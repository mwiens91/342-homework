% Set up the document
\documentclass{article}

% Page size
\usepackage[
    letterpaper,]{geometry}

% Lines between paragraphs
\setlength{\parskip}{\baselineskip}
\setlength{\parindent}{0pt}

% Math
\usepackage{mathtools}
\usepackage{amssymb}
\usepackage{amsthm}
\usepackage{commath}

% Number sets
\newcommand{\C}{\mathcal{C}}
\newcommand{\N}{\mathbb{N}}
\newcommand{\Q}{\mathbb{Q}}
\newcommand{\R}{\mathbb{R}}
\newcommand{\Z}{\mathbb{Z}}

% Links
\usepackage{hyperref}

% Page numbers at top right
\usepackage{fancyhdr}
\pagestyle{fancy}
\fancyhf{}
\fancyhead[R]{\thepage}
\renewcommand\headrulewidth{0pt}

\begin{document}

\textbf{MATH 342 assignment 3} \\
\textbf{Matt Wiens \#301294492} \\
\textbf{2020-06-15}

1. Find an inverse modulo $19$ for each of the following integers.

(a) 6

\textit{Solution.}
We want to find $x \in \Z$ such that $6 x \equiv 1 \mod 19$. Since
$\gcd(6, 19) = 1$, there is exactly one solution (modulo $19$).
By testing values for $x$ from $0$ to $18$, we find that
$6 \cdot 16 \equiv 1 \mod 9$, so we can conclude that the inverse
of $6$ modulo $19$ is $16$.

\vspace{5mm}

(b) 9

\textit{Solution.}
Again, we want to find an $x \in \Z$ such that $9 x \equiv 1 \mod 19$.
Since $\gcd(9, 19) = 1$, we again have that there is one solution modulo
$19$. Again testing for values of $x$ from $0$ to $18$ we find that $6
\cdot 17 \equiv 1 \mod 9$. Thus the inverse of $9$ modulo $19$ is $17$.


\newpage

2. What is the last digit in the ordinary decimal representation of
$3^{400}$?

\textit{Solution.}
First note that $\phi(10) = \#\cbr{1, 3, 7, 9} = 4$.

We want to find $3^{400} \mod 10$. We can solve this by noting that,
using Euler's Theorem,
%
\begin{equation*}
    3^{400} \equiv 3^{4 (100)} \equiv 1 \mod 10
    .
\end{equation*}
%
Thus the last digit of $3^{400}$ is $1$.

\newpage

3. What are the last two digits in the ordinary decimal representation
of $3^{400}$?

\textit{Solution.}
First we'll use Euler's product formula to determine $\phi(100)$.
Since the prime divisors of $100$ are $2$ and $5$, we have that
%
\begin{equation*}
    \phi(100) = 100 \del{1 - \frac{1}{2}} \del{1 - \frac{1}{5}} = 40
\end{equation*}

Using that, we want to find $3^{400} \mod 100$. We can solve this again
using Euler's Theorem as follows:
%
\begin{equation*}
    3^{400} \equiv 3^{40 (10)} 4 \equiv 1 \mod 100
    .
\end{equation*}
%
Thus the last two digits of $3^{400}$ are $01$.

\newpage

4. Find all the solutions of each of the following systems of linear
congruences.

(a) $x \equiv 1 \mod 2$, $x \equiv 2 \mod 3$

\textit{Solution.}
Using the notation in the Chinese Remainder Theorem, we have
$a_1 = 1, a_2 = 2, m_1 = 2, m_2 = 3$ and hence
$M_1 = m_2 = 3, M_2 = m_1 = 2$.

Now we need to solve two modular inverses, the first of which
is $3 y_1 \equiv 1 \mod 2$, which has solution $y_1 = 1$.
The other inverse we need to solve is $2 y_2 \equiv 1 \mod 3$
which has solution $y_2 = 2$.

Hence the unique solution is given by
%
\begin{equation*}
    x
    \equiv a_1 M_1 y_1 + a_2 M_2 y_2
    \equiv 1 \cdot 3 \cdot 1 + 2 \cdot 2 \cdot 2
    \equiv 11
    \equiv 5
    \mod 6
    .
\end{equation*}

\vspace{5mm}

(b) $x \equiv 1 \mod 2$, $x \equiv 2 \mod 3$, $x \equiv 3 \mod 5$

\textit{Solution.}
Using the notation in the Chinese Remainder Theorem, we have
$a_1 = 1, a_2 = 2, a_3 = 3, m_1 = 2, m_2 = 3, m_3 = 5$ and hence
$M_1 = 15, M_2 = 5, M_3 = 6$.

Now we need to solve three modular inverses, the first of which
is $15 y_1 \equiv 1 \mod 2$, which has solution $y_1 = 1$.
The second inverse we need to solve is $5 y_2 \equiv 1 \mod 3$ which has
solution $y_2 = 2$. The final inverse is $6 y_3 \equiv 1 \mod 5$ which
has solution $y_3 = 1$.

Hence the unique solution is given by
%
\begin{equation*}
    x
    \equiv a_1 M_1 y_1 + a_2 M_2 y_2 + a_3 M_3 y_3
    \equiv 1 \cdot 15 \cdot 1 + 2 \cdot 5 \cdot 2 + 3 \cdot 6 \cdot 1
    \equiv 53
    \equiv 23
    \mod 30
    .
\end{equation*}

\vspace{5mm}

(c) $3x \equiv 4 \mod 8$

\textit{Solution.}
Because $\gcd\del{3, 8} = 1 | 4$, there is exactly one solution modulo $8$
to this congruence. Testing all values of $x$ from $0$ to $7$ we have
that $x \equiv 4 \mod 8$ is the unique solution.

\newpage

5. Find all solutions of $x^2 + x + 3 \equiv 0 \mod 144$

\textit{Solution.}
First we'll express
%
\begin{equation*}
    x^2 + x + 3 \equiv \del{x + \frac{1}{2}}^2 + \frac{11}{4} \equiv 0 \mod 144
    .
\end{equation*}
%
Letting $y = x + 1/2$ we have
%
\begin{equation*}
    y^2 \equiv - \frac{11}{4} \mod 144
    .
\end{equation*}
%
Since there is no rational solution for $y$ (while there is an
irrational solution), it follows that $x$ is also irrational, and hence
no solutions exist for the original congruence.

\end{document}
