% Set up the document
\documentclass{article}

% Page size
\usepackage[
    letterpaper,]{geometry}

% Lines between paragraphs
\setlength{\parskip}{\baselineskip}
\setlength{\parindent}{0pt}

% Math
\usepackage{mathtools}
\usepackage{amssymb}
\usepackage{amsthm}
\usepackage{commath}

% Operators
\newcommand{\Mod}[1]{\ (\mathrm{mod}\ #1)}
\DeclareMathOperator{\ord}{ord}

% Number sets
\newcommand{\C}{\mathcal{C}}
\newcommand{\N}{\mathbb{N}}
\newcommand{\Q}{\mathbb{Q}}
\newcommand{\R}{\mathbb{R}}
\newcommand{\Z}{\mathbb{Z}}

% Links
\usepackage{hyperref}

% Page numbers at top right
\usepackage{fancyhdr}
\pagestyle{fancy}
\fancyhf{}
\fancyhead[R]{\thepage}
\renewcommand\headrulewidth{0pt}

\begin{document}

\textbf{MATH 342 assignment 6} \\
\textbf{Matt Wiens \#301294492} \\
\textbf{2020-07-27}

1. Find the incongruent roots modulo $11$ of

(a) $x^2 + 2$

\textit{Solution.}

\vspace{5mm}

(b) $x^4 + x^2 + 1$

\textit{Solution.}

\newpage

2. Find a primitive root modulo $11^2$.

\textit{Solution.}

\newpage

3. Show that if $a$ and $n$ are relatively prime integers with $n > 1$,
then $n$ is prime if and only if $(x - a)^n$ and $x^n - a$ are congruent
modulo $n$ as polynomials.

\textit{Solution.}
(See hint in homework sheet)

\newpage

4. Let $a$ and $n > 1$ be any integers such that $a^{n - 1} \equiv 1 \Mod{n}$
but $a^d \not\equiv 1 \Mod{n}$ for every proper divisor $d$ of $n - 1$. Prove
that $n$ is a prime.

\textit{Solution.}

\newpage

5. Let $a$ be a primitive root of $p$ with $p \equiv 1 \Mod{4}$. Show
that $-a$ is also a primitive root.

\textit{Solution.}

\newpage

6. Show that there are the same number of primitive roots modulo $2p^a$
as there are modulo $p^a$, where $p$ is an odd prime and $a$ is a
positive integer.

\textit{Solution.}

\newpage

7. Show that the integer $n$ has a primitive root if and only if the
only solutions of the congruence $x^2 \equiv 1 \Mod{n}$ are
$x \equiv \pm 1 \Mod{n}$.

\textit{Solution.}

\newpage

8. Let $p$ be a prime. If for integers $k$ and $l$ we have
$x^k \equiv x^l \Mod{p}$ for all $x \in \Z$ with $(x, p) = 1$,
show that $k \equiv l \Mod{p - 1}$.

\textit{Solution.}

\newpage

9. Show that if $n$ is a positive integer, and $a$ and $b$ are integers
relatively prime to $n$ such that $(\ord_n a, \ord_n b) = 1$, then
$\ord_n ab = \ord_n a \cdot \ord_n b$.

\textit{Solution.}

\newpage

10. Show that if $n$ is a positive integer, and $a$ is relatively prime
to $n$, then $\ord_n \bar{a} = \ord_n a$, where $\bar{a}$ is the inverse
of $a$ modulo $n$.

\textit{Solution.}

\newpage

11. (a) Let $k$ and $a$ be positive integers, with $a \geq 2$. Show
that $k \mid \phi(a^k - 1)$.

\textit{Solution.}

\vspace{5mm}

(b) Let $p$ be a prime number. Show that if $p \mid \phi(m)$ and
$p \nmid m$, then there is at least one prime factor $q$ of $m$
such that $q \equiv 1 \Mod{p}$.

\textit{Solution.}

\vspace{5mm}

(c) Let $p$ be a given prime number. Prove that there exists infinitely
many prime numbers $q \equiv 1 \Mod{p}$.

\textit{Solution.}

\newpage

12. For which positive integers $a$ is the congruence
$a x^4 \equiv 2 \Mod{13}$ solvable?

\textit{Solution.}

\end{document}
