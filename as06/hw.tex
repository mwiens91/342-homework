% Set up the document
\documentclass{article}

% Page size
\usepackage[
    letterpaper,]{geometry}

% Lines between paragraphs
\setlength{\parskip}{\baselineskip}
\setlength{\parindent}{0pt}

% Math
\usepackage{mathtools}
\usepackage{amssymb}
\usepackage{amsthm}
\usepackage{commath}

% Operators
\newcommand{\Mod}[1]{\ (\mathrm{mod}\ #1)}
\DeclareMathOperator{\ord}{ord}

% Number sets
\newcommand{\C}{\mathcal{C}}
\newcommand{\N}{\mathbb{N}}
\newcommand{\Q}{\mathbb{Q}}
\newcommand{\R}{\mathbb{R}}
\newcommand{\Z}{\mathbb{Z}}

% Links
\usepackage{hyperref}

% Page numbers at top right
\usepackage{fancyhdr}
\pagestyle{fancy}
\fancyhf{}
\fancyhead[R]{\thepage}
\renewcommand\headrulewidth{0pt}

\begin{document}

\textbf{MATH 342 assignment 6} \\
\textbf{Matt Wiens \#301294492} \\
\textbf{2020-07-27}

1. Find the incongruent roots modulo $11$ of

(a) $x^2 + 2$

\textit{Solution.}
I'm not sure what the intended strategy is here, but the modulus $11$ is
small enough that we can test all $x = 0, \ldots, 10$ to see that
both $x \equiv 3 \Mod{11}$ and $x \equiv 8 \Mod{11}$ satisfy
$x^2 + 2 \equiv 0 \Mod{11}$.

\vspace{5mm}

(b) $x^4 + x^2 + 1$

\textit{Solution.}
Again we can use brute force since the modulus $11$ is small and test
all $x = 0, \ldots, 10$, which this time tells us that
$x^4 + x^2 + 1 \equiv 0 \Mod{11}$ has no solutions.

\newpage

2. Find a primitive root modulo $11^2$.

\textit{Solution.}
We first find a primitive root modulo $11$, meaning we want to find an $a$
such that $\ord_{11} a = \phi(11) = 10$. Testing $a = 2$ we have
%
\begin{align*}
    2^2 &\equiv 4 \Mod{11}, \\
    2^5 &\equiv 32 \equiv 10 \Mod{11}, \\
    2^{10} &\equiv 1024 \equiv 1 \Mod{11}
    .
\end{align*}
%
Hence we see that $\ord_{11} 2 = \phi(11)$, and so we have found a
primitive root modulo $11$. To find a primitive root modulo $11^2$,
we first note that $\phi(11^2) = 11 \cdot 10 = 110$. We now test
both $a = 2, 14$ for being primitive roots modulo $11^2$. For
$a = 2$ we have
%
\begin{align*}
    2^{10} &\equiv 1024 \equiv 56 \not\equiv \Mod{11^2}
    , \\
    2^{110} &\equiv 1298074214633706907132624082305024 \equiv 1 \Mod{11^2}
    .
\end{align*}
%
Hence by the theorem in class, we have found that
$\ord_{11^2} 2 = \phi(11^2) = 110$, and so $2$ is a primitive root modulo
$11^2$ (there is no need to check $a = 14$ since we have already found a
root).


\newpage

3. Show that if $a$ and $n$ are relatively prime integers with $n > 1$,
then $n$ is prime if and only if $(x - a)^n$ and $x^n - a$ are congruent
modulo $n$ as polynomials.

\textit{Solution.}
We want to show that $n$ is prime if and only if
%
\begin{align*}
    (x - a)^n
        &\equiv \sum_{k = 0}^n \binom{n}{k} x^{n - k} (-a)^k \\
        &\equiv x^n - a + \sum_{k = 1}^{n - 1} \binom{n}{k} x^{n - k} (-a)^k \\
        &\equiv x^n - a
        \Mod{n}
        .
\end{align*}
%
Stated another way, we need to show that $n$ is prime if and only if
$\sum_{k = 1}^{n - 1} \binom{n}{k} x^{n - k} (-a)^k \equiv 0 \Mod{n}$.
Since when $n$ is prime $n \mid \binom{n}{k}$, we have
$\sum_{k = 1}^{n - 1} \binom{n}{k} x^{n - k} (-a)^k \equiv 0 \Mod{n}$;
if $n$ is not prime, then $n \nmid \binom{n}{k}$ (and also $n \nmid a$),
so, as a polynomial, we have
$\sum_{k = 1}^{n - 1} \binom{n}{k} x^{n - k} (-a)^k \not\equiv 0 \Mod{n}$.

\newpage

4. Let $a$ and $n > 1$ be any integers such that $a^{n - 1} \equiv 1 \Mod{n}$
but $a^d \not\equiv 1 \Mod{n}$ for every proper divisor $d$ of $n - 1$. Prove
that $n$ is a prime.

\textit{Solution.}
Suppose that $n$ is composite. We know that $a^{\phi(n)} \equiv 1 \Mod n$.
Since $\ord_{n} a \leq \phi(n) < n - 1$, we have that $\ord_{n} a < n - 1$
and $\ord_{n} a \mid n - 1$. However, this contradicts our assumption
that $a^d \not\equiv 1 \Mod{n}$ for every proper divisor $d \mid n - 1$.
Hence $n$ cannot be composite, and must be prime.

\newpage

5. Let $a$ be a primitive root of $p$ with $p \equiv 1 \Mod{4}$. Show
that $-a$ is also a primitive root.

\textit{Solution.}
Let $k$ be such that $p = 1 + 4k$. Since $p$ is prime, we have
$\phi(p) = 4k$, and so we have $\ord_{p} a = 4k$. Hence
$a^{4k} \equiv 1 \Mod{p}$. Since $\ord_{p} a = 4k$, and expressing
$a^{4k} = (a^{2k})^2$,  we must have that $a^{2k} \equiv - 1 \Mod{p}$,
which, because $2k$ is an even power, we are free to write as
$(-a)^{2k} \equiv - 1 \Mod{p}$. Multiplying through by $- a$, we have
$(-a)^{2k + 1} \equiv a \Mod{p}$.

Using a slightly modified version of a theorem from lectures, we have
that $a$ is a primitive root of $p$ if and only if
$a^1, a^2, \ldots, a^{p - 1}$ form a reduced residue set modulo $n$.
Since $a \equiv (-a)^{2k + 1} \Mod{p}$, we also have that
$(-a), (-a)^2, \ldots, (-a)^{p - 1}$ form a reduced residue set modulo
$n$, and hence $-a$ is also a primitive root of $p$.

\newpage

6. Show that there are the same number of primitive roots modulo $2p^a$
as there are modulo $p^a$, where $p$ is an odd prime and $a$ is a
positive integer.

\textit{Solution.}
This is trivial to show. Since
%
\begin{equation*}
    \phi(2 p^a) = \phi(2) \phi(p^a) = \phi(p^a)
    ,
\end{equation*}
%
Then we have that, using $F(d)$ as in lectures to denote the number of
integers less than $p$ that have order $d$ modulo $p$,
%
\begin{equation*}
    F(\phi(2 p^a)) = F(\phi(p^a))
    .
\end{equation*}
%
Hence there are the same number of primitive roots modulo $2p^a$
as there are modulo $p^a$, provided that $p$ is an odd prime.

\newpage

7. Show that the integer $n$ has a primitive root if and only if the
only solutions of the congruence $x^2 \equiv 1 \Mod{n}$ are
$x \equiv \pm 1 \Mod{n}$.

\textit{Solution.}
Fix $n$ and suppose the only solutions of $x^2 \equiv 1 \Mod{n}$ are
$x \equiv \pm 1 \Mod{n}$. Suppose for contradiction that $n$
has no primitive root. Then for any prime $p$, we know that
$n \neq 2, 4, p^a, 2p^a$. Hence either $n = 2^a$ for some $a \geq 3$, or
$n = c d$, where $c$ and $d$ can be made coprime. If $n = 2^a$, consider
that $x_0 \equiv 2^{a - 1} - 1 \not\equiv \pm 1 \Mod{2^a}$, but
%
\begin{equation*}
    x_0^2
    \equiv (2^{a -1} - 1)^2
    \equiv 2^{a - 1} \cdot 2^{a - 1} - 2 \cdot 2^{a - 1} + 1
    \equiv 1
    \Mod{2^a}
    .
\end{equation*}
%
Hence we have found a solution $x_0 \neq \pm 1 \Mod{n}$ that satisfies
$x^2 \equiv 1 \Mod{n}$, which is a contradiction. Hence $n \neq 2^a$,
so instead we must have that $n = c d$ with $c, d$ coprime. Since
$x \equiv 1 \Mod{c}$ and $x \equiv 1 \Mod{d}$ each have at least two solutions,
by the Chinese remainder theorem $x \equiv 1 \Mod{c d}$ has at least four solutions.
This again results in a contradiction and hence we cannot have $n = c d$ with
$c$ and $d$ coprime. Thus $n$ must have a primitive root.

Suppose $n$ has a primitive root, say, $a$. Then $a^1, a^2, \ldots, a^{\phi(n)}$
form a reduced residue set modulo $n$. Hence solutions must have the form
$x_0 \equiv a^{k_1}$ for some $k_1$ to $x^2 \equiv 1 \Mod{n}$; that is, for some $k_1$
we have $a^{2k_1} \equiv 1 \Mod{n}$. Since $n$ has a primitive root, $\ord_n a = \phi(n)$
and therefore going back to the previous equation we have $\phi(n) \mid 2 k_1$.
Thus for some $k_2$ we have $2 k_1 = k_2 \phi(n)$, and therefore we have,
using that by a theorem in lectures, $a^{\phi(n)/2} = -1$,
%
\begin{equation*}
    x_0 \equiv a^{k_1} \equiv a^{k_2 \phi(n) / 2} \equiv (-1)^{k_2} \equiv \pm 1 \Mod{n}
    .
\end{equation*}
%
Hence the only solutions to the congruence $x^2 \equiv 1 \Mod{n}$ are $x \equiv \pm 1 \Mod{n}$.

\newpage

8. Let $p$ be a prime. If for integers $k$ and $l$ we have
$x^k \equiv x^l \Mod{p}$ for all $x \in \Z$ with $(x, p) = 1$,
show that $k \equiv l \Mod{p - 1}$.

\textit{Solution.}
From lectures we have that if $a$ and $n$ are relatively prime integers
with $n > 0$, then $a^i \equiv a^j \Mod{n}$ is equivalent to
$i \equiv j \Mod{\ord_n a}$. Here, because we have $x^k \equiv x^l \Mod{p}$,
with $x$ and $p$ relatively prime integers, we have that
$k \equiv l \Mod{\ord_p x}$. Hence we need to show that $\ord_p x = \phi(p) = p - 1$,
or equivalently, that $x$ is a primitive root modulo $p$. However
this is true because the correspondence between $k$ and $l$ guarantees
that $x^1, x^2, \ldots, x^{p - 1}$ form a reduced residue set modulo $p$.

\newpage

\end{document}
