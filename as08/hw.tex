% Set up the document
\documentclass{article}

% Page size
\usepackage[
    letterpaper,]{geometry}

% Lines between paragraphs
\setlength{\parskip}{\baselineskip}
\setlength{\parindent}{0pt}

% Math
\usepackage{mathtools}
\usepackage{amssymb}
\usepackage{amsthm}
\usepackage{commath}

% Operators
\newcommand{\Mod}[1]{\ (\mathrm{mod}\ #1)}
\newcommand{\Leg}[2]{\del{\frac{#1}{#2}}}
\DeclareMathOperator{\ord}{ord}
\DeclarePairedDelimiter\ceil{\lceil}{\rceil}
\DeclarePairedDelimiter\floor{\lfloor}{\rfloor}

% Number sets
\newcommand{\C}{\mathcal{C}}
\newcommand{\N}{\mathbb{N}}
\newcommand{\Q}{\mathbb{Q}}
\newcommand{\R}{\mathbb{R}}
\newcommand{\Z}{\mathbb{Z}}

% Links
\usepackage{hyperref}

% Page numbers at top right
\usepackage{fancyhdr}
\pagestyle{fancy}
\fancyhf{}
\fancyhead[R]{\thepage}
\renewcommand\headrulewidth{0pt}

\begin{document}

\textbf{MATH 342 assignment 8} \\
\textbf{Matt Wiens \#301294492} \\
\textbf{2020-08-12}

1. Evaluate the Legendre symbol $\Leg{13}{23}$ using

(a) Euler's criterion

\textit{Solution.}
Using Euler's criteron we have that
%
\begin{equation*}
    \Leg{13}{23}
    \equiv 13^{(23 - 1) / 2}
    \equiv 13^{11}
    \equiv 1
    \Mod{23}
    ,
\end{equation*}
%
and hence
%
\begin{equation*}
    \Leg{13}{23}
    = 1
    .
\end{equation*}

\vspace{5mm}

(b) Gauss's lemma

\textit{Solution.}
Now we will compute the same result using Gauss's lemma.
Since
%
\begin{align*}
    1 \cdot 13 &\equiv 13 \Mod{23}, \\
    2 \cdot 13 &\equiv 3 \Mod{23}, \\
    3 \cdot 13 &\equiv 16 \Mod{23}, \\
    4 \cdot 13 &\equiv 6 \Mod{23}, \\
    5 \cdot 13 &\equiv 19 \Mod{23}, \\
    6 \cdot 13 &\equiv 9 \Mod{23}, \\
    7 \cdot 13 &\equiv 22 \Mod{23}, \\
    8 \cdot 13 &\equiv 12 \Mod{23}, \\
    9 \cdot 13 &\equiv 2 \Mod{23}, \\
    10 \cdot 13 &\equiv 15 \Mod{23}, \\
    11 \cdot 13 &\equiv 5 \Mod{23},
\end{align*}
%
we have that, using the notation of Gauss's lemma, $s = 6$,
and hence
%
\begin{equation*}
    \Leg{13}{23}
    = (-1)^{6}
    = 1
    .
\end{equation*}

\newpage

2. Using the Quadratic Reciprocity Law, show that
%
\begin{equation*}
    \Leg{3}{p}
    = (-1)^{\floor*{\frac{p + 1}{6}}}
    =
    \begin{cases}
        1, & p \equiv 1, 11 \Mod{12}, \\
        -1, & p \equiv 5, 7 \Mod{12}.
    \end{cases}
\end{equation*}

\textit{Solution.}
In order for this Legendre symbol to make sense, we need $p$ to be an odd prime
and for $p \nmid 3$ (so $p \neq 3$). Hence, modulo $12$, we have
$p$ an odd prime with either $p \equiv 1, 5, 7, 11 \Mod{12}$.
For $p \equiv 1 \Mod{12}$, then we have
%
\begin{align*}
    \Leg{3}{p}
    &= (-1)^{\frac{p - 1}{2} \cdot \frac{3 - 1}{2}} \Leg{p}{3} \\
    &= (-1)^{\frac{(12 k + 1) - 1}{2} \cdot 1} \Leg{p}{3} \\
    &= (-1)^{6 k} \Leg{p}{3} \\
    &= \Leg{p}{3} \\
    &= \Leg{1}{3} \\
    &= 1
    .
\end{align*}
%
and hence for $p \equiv 1 \Mod{12}$, we have
%
\begin{equation*}
    \Leg{3}{p}
    = 1
    = (-1)^{2 k}
    = (-1)^{\floor*{\frac{12 k + 2}{6}}}
    = (-1)^{\floor*{\frac{(12 k + 1) + 1}{6}}}
    = (-1)^{\floor*{\frac{p + 1}{6}}}
    .
\end{equation*}
%
Now consider when $p \equiv 11 \Mod{12}$. Then
%
\begin{align*}
    \Leg{3}{p}
    &= (-1)^{\frac{p - 1}{2} \cdot \frac{3 - 1}{2}} \Leg{p}{3} \\
    &= (-1)^{\frac{(12 k + 11) - 1}{2} \cdot 1} \Leg{p}{3} \\
    &= (-1)^{6 k + 5} \Leg{p}{3} \\
    &= - \Leg{p}{3} \\
    &= - \Leg{2}{3} \\
    &= - (-1) \\
    &= 1
    .
\end{align*}
%
Hence for $p \equiv 11 \Mod{12}$ we have
%
\begin{equation*}
    \Leg{3}{p}
    = 1
    = (-1)^{2 k + 2}
    = (-1)^{\floor*{\frac{12 k + 12}{6}}}
    = (-1)^{\floor*{\frac{(12 k + 11) + 1}{6}}}
    = (-1)^{\floor*{\frac{p + 1}{6}}}
    .
\end{equation*}
%
Now, consider when $p \equiv 5 \Mod{12}$. Then we have
%
\begin{align*}
    \Leg{3}{p}
    &= (-1)^{\frac{p - 1}{2} \cdot \frac{3 - 1}{2}} \Leg{p}{3} \\
    &= (-1)^{\frac{(12 k + 5) - 1}{2} \cdot 1} \Leg{p}{3} \\
    &= (-1)^{6 k + 4} \Leg{p}{3} \\
    &= \Leg{p}{3} \\
    &= \Leg{2}{3} \\
    &= -1
    .
\end{align*}
%
Hence for $p \equiv 5 \Mod{12}$ we have
%
\begin{equation*}
    \Leg{3}{p}
    = - 1
    = (-1)^{2 k + 1}
    = (-1)^{\floor*{\frac{12 k + 6}{6}}}
    = (-1)^{\floor*{\frac{(12 k + 5) + 1}{6}}}
    = (-1)^{\floor*{\frac{p + 1}{6}}}
    .
\end{equation*}
%
Finally, for $p \equiv 7 \Mod{12}$, we have
%
\begin{align*}
    \Leg{3}{p}
    &= (-1)^{\frac{p - 1}{2} \cdot \frac{3 - 1}{2}} \Leg{p}{3} \\
    &= (-1)^{\frac{(12 k + 7) - 1}{2} \cdot 1} \Leg{p}{3} \\
    &= (-1)^{6 k + 3} \Leg{p}{3} \\
    &= - \Leg{p}{3} \\
    &= - \Leg{1}{3} \\
    &= -1
    .
\end{align*}
%
Hence for $p \equiv 7 \Mod{12}$ we have
%
\begin{equation*}
    \Leg{3}{p}
    = - 1
    = (-1)^{2 k + 1}
    = (-1)^{\floor*{\frac{12 k + 8}{6}}}
    = (-1)^{\floor*{\frac{(12 k + 7) + 1}{6}}}
    = (-1)^{\floor*{\frac{p + 1}{6}}}
    .
\end{equation*}
%
To summarize, we have shown that
%
\begin{equation*}
    \Leg{3}{p}
    = (-1)^{\floor*{\frac{p + 1}{6}}}
    =
    \begin{cases}
        1, & p \equiv 1, 11 \Mod{12}, \\
        -1, & p \equiv 5, 7 \Mod{12}.
    \end{cases}
\end{equation*}


\newpage

4. For any prime $p$ of the form $4 k + 3$, prove that
$x^2 + \frac{p + 1}{4} \equiv 0 \Mod{p}$ has no solution.

\textit{Solution.}
Suppose $p = 4 k + 3$, then $(p + 3) / 4 = k + 1$. Suppose
for contradiction that there exists $x_0$ that solves
$x^2 = - (p + 1) / 4 = - k - 1 \Mod{p}$. Then we have
%
\begin{align*}
    x_0^{p - 1}
    &\equiv x_0^{(4k + 3) - 1} \\
    &\equiv x_0^{4k + 2} \\
    &\equiv x_0^2 ((x_0^2)^2)^k \\
    &\equiv (-k - 1) (k^2 + 2k + 1)^k
    \Mod{p}
\end{align*}
%
For no $k$ do we have $(-k - 1) (k^2 + 2k + 1)^k \equiv 1 \Mod{p}$
(this is because the first term is always negative, and the second
term is always positive), so the above contradicts Fermat's Little Theorem.

Hence no solution $x_0$ can exist to $x^2 + \frac{p + 1}{4} \equiv 0 \Mod{p}$.

\newpage

9. For which primes are there solutions $x^2 + y^2 \equiv 0 \Mod{p}$
with $(x, p) = (y, p) = 1$?

\textit{Solution.}
Suppose such a solution exists for some odd prime $p$. Since we have that
$p \nmid y$, there exists an inverse of $y$ modulo $p$. Denote this
inverse by $z$. Hence we have that
%
\begin{align*}
    &x^2 + y^2 \equiv 0 \Mod{p} \\
    &\iff x^2 \equiv - y^2 \Mod{p} \\
    &\iff x^2 z^2 \equiv - 1 \Mod{p}, \\
    &\iff (xz)^2 \equiv - 1 \Mod{p}.
\end{align*}
%
Note that since $(y, p) = 1$, we must have $(z, p) = 1$ and thus
$p \nmid xz$. Hence we can apply Fermat's Little Theorem to get
%
\begin{align*}
    &(x z)^{p - 1} \equiv 1 \Mod{p} \\
    &\iff ((x z)^2)^{(p - 1) / 2} \equiv 1 \Mod{p} \\
    &\iff (-1)^{(p - 1) / 2} \equiv 1 \Mod{p}.
\end{align*}
%
Hence we must have that
%
\begin{align*}
    &(p - 1) / 2 \equiv 0 \Mod{2} \\
    &\implies p - 1 \equiv 0 \Mod{4} \\
    &\implies p \equiv 1 \Mod{4}.
\end{align*}
%
Hence only for odd primes which are congruent to $1$ modulo $4$,
can $x^2 + y^2 \equiv 0 \Mod{p}$ have a solution.

Now we will show that there does exist such a solution for every
$p \equiv 1 \Mod{4}$. Since $p$ is congruent to $1$ modulo $4$,
from lecture we know that $-1$ is a quadratic residue modulo $p$.
Hence there exists $x_0$ such that $x_0^2 \equiv - 1 \Mod{p}$;
but this is equivalent to $x_0^2 + 1^2 \equiv 0 \Mod{p}$,
which is a solution since $(x_0, p) = (1, p) = 1$.

Hence we have shown that for odd primes $p$, there are solutions to
$x^2 + y^2 \equiv 0 \Mod{p}$ (with $x$ and $y$ relatively prime to $p$)
if and only if $p \equiv 1 \Mod{4}$.

Note that for $p = 2$, there is clearly a solution since $1^2 + 1^2 = 0 \Mod{2}$
with $(1, 2) = 1$.

\newpage

10. For which prime powers are there solutions $x^2 + y^2 \equiv 0 \Mod{p^\alpha}$
with $(x, p) = (y, p) = 1$?

\textit{Solution.}
Using question 9, we know that for $\alpha = 1$, there are solutions for $p = 2$
or $p \equiv 1 \Mod{4}$.

If $x^2 + y^2 = 0 \Mod{p}$,
we have $(x^2 + y^2)(x^2 + y^2) \equiv (x^2 - y^2)^2 + (2xy)^2 \equiv 0 \Mod{p^2}$.
Hence in a sense we can ``lift'' our solution from $p^1$ to $p^2$. Suppose
$x_k$ and $y_k$ solves $x_k^2 + y_k^2 \equiv 0 \Mod{p^k}$. Then
%
\begin{align*}
    x_{k + 1} &\coloneqq x_k x - y_k y, \\
    y_{k + 1} &\coloneqq x_k y + y_k x
\end{align*}
%
solves $x_{k + 1}^2 + y_{k + 1}^2 \equiv 0 \Mod{p^{k + 1}}$, since
%
\begin{equation*}
    (x_k^2 + y_k^2) (x^2 + y^2) \equiv (x_kx - y_k y)^2 + (x_k y + y_k x)^2 \equiv 0 \Mod{p^{k + 1}}
    .
\end{equation*}
%
Hence, because for odd primes $p \equiv 1 \Mod{4}$ we have that for $x$ and $y$ solving
$x^2 + y^2 \equiv 0 \Mod{p}$ we can choose $x \neq y$, we can lift solutions up
non-trivially to any $p^\alpha$. This does not work for $p = 2$ however,
since the solutions $x = y = 1$, which would yield a trivial solution if we tried
to lift it.

I am unsure if for $p \equiv 3 \Mod{4}$ there are solutions for $\alpha > 1$.

\newpage

11. Evaluate each of the following Jacobi or Legendre symbols:

(a) $\Leg{111}{1001}$

\textit{Solution.}
Here we have $111 = 3 \cdot 37$ and $1001 = 7 \cdot 11 \cdot 13$.
Hence $111$ and $1001$ are relatively prime and we can use that
%
\begin{align*}
    \Leg{111}{1001}
    &= (-1)^{\frac{111 - 1}{2} \cdot \frac{1001 - 1}{2}} \Leg{1001}{111} \\
    &= (-1)^{55 \cdot 500} \Leg{1001}{111} \\
    &= \Leg{1001}{111} \\
    &= \Leg{2}{111}
    .
\end{align*}
%
Now we will use the definition of the Jacobi symbol to write
%
\begin{equation*}
    \Leg{111}{1001} = \Leg{2}{111} = \Leg{2}{3} \Leg{3}{37}
    .
\end{equation*}
%
Using Euler's Criterion, we have
%
\begin{equation*}
    \Leg{2}{3} \equiv 2^{\frac{3 - 1}{2}} \equiv 2^1 \equiv -1 \Mod{3}
    ,
\end{equation*}
%
and thus $\Leg{2}{3} = -1$. For $\Leg{3}{37}$, we first simplify using
quadratic reciprocity:
%
\begin{align*}
    \Leg{3}{37}
    &= (-1)^{\frac{3 - 1}{2} \cdot \frac{37 - 1}{2}} \Leg{37}{3} \\
    &= (-1)^{1 \cdot 18} \Leg{37}{3} \\
    &= \Leg{37}{3} \\
    &= \Leg{1}{3}
    .
\end{align*}
%
Hence $\Leg{3}{37} = \Leg{1}{3} = 1$. Thus we have that
%
\begin{equation*}
    \Leg{111}{1001} = \Leg{2}{3} \Leg{3}{37} = (-1) \cdot 1 = -1
    .
\end{equation*}

\vspace{5mm}

(d) $\Leg{105}{1009}$

\textit{Solution.}
Note that $105 = 3 \cdot 5 \cdot 7$ and that $1009$ is prime.
Hence we can use that
%
\begin{equation*}
    \Leg{105}{1009} = \Leg{3}{1009} \Leg{5}{1009} \Leg{7}{1009}
    .
\end{equation*}
%
Evaluating each of terms on the right hand side of the above equation separately,
we have
%
\begin{align*}
    \Leg{3}{1009}
    &= (-1)^{\frac{1009 - 1}{2} \cdot \frac{3 - 1}{2}} \Leg{1009}{3} \\
    &= (-1)^{504 \cdot 1} \Leg{1009}{3} \\
    &= \Leg{1009}{3} \\
    &= \Leg{1}{3}, \\
    %
    \Leg{5}{1009}
    &= (-1)^{\frac{1009 - 1}{2} \cdot \frac{5 - 1}{2}} \Leg{1009}{5} \\
    &= (-1)^{504 \cdot 2} \Leg{1009}{5} \\
    &= \Leg{1009}{5} \\
    &= \Leg{4}{5}, \\
    %
    \Leg{7}{1009}
    &= (-1)^{\frac{1009 - 1}{2} \cdot \frac{7 - 1}{2}} \Leg{1009}{7} \\
    &= (-1)^{504 \cdot 3} \Leg{1009}{7} \\
    &= \Leg{1009}{7} \\
    &= \Leg{1}{7}.
\end{align*}
%
Clearly we have $\Leg{3}{1009} = \Leg{1}{3} = 1$ and $\Leg{7}{1009} = \Leg{1}{7} = 1$.
To determine the remaining term, we use Euler's Criterion:
%
\begin{equation*}
    \Leg{5}{1009} = \Leg{4}{5} \equiv 4^{\frac{5 - 1}{2}} \equiv 4^2 \equiv 16 \equiv 1 \Mod{5}
    .
\end{equation*}
%
Hence $\Leg{5}{1009} = 1$. Thus, putting all of our results together we have
%
\begin{equation*}
    \Leg{105}{1009}
    = \Leg{3}{1009} \Leg{5}{1009} \Leg{7}{1009}
    = 1 \cdot 1 \cdot 1
    = 1
    .
\end{equation*}

\end{document}

