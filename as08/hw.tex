% Set up the document
\documentclass{article}

% Page size
\usepackage[
    letterpaper,]{geometry}

% Lines between paragraphs
\setlength{\parskip}{\baselineskip}
\setlength{\parindent}{0pt}

% Math
\usepackage{mathtools}
\usepackage{amssymb}
\usepackage{amsthm}
\usepackage{commath}

% Operators
\newcommand{\Mod}[1]{\ (\mathrm{mod}\ #1)}
\newcommand{\Leg}[2]{\del{\frac{#1}{#2}}}
\DeclareMathOperator{\ord}{ord}
\DeclarePairedDelimiter\ceil{\lceil}{\rceil}
\DeclarePairedDelimiter\floor{\lfloor}{\rfloor}

% Number sets
\newcommand{\C}{\mathcal{C}}
\newcommand{\N}{\mathbb{N}}
\newcommand{\Q}{\mathbb{Q}}
\newcommand{\R}{\mathbb{R}}
\newcommand{\Z}{\mathbb{Z}}

% Links
\usepackage{hyperref}

% Page numbers at top right
\usepackage{fancyhdr}
\pagestyle{fancy}
\fancyhf{}
\fancyhead[R]{\thepage}
\renewcommand\headrulewidth{0pt}

\begin{document}

\textbf{MATH 342 assignment 8} \\
\textbf{Matt Wiens \#301294492} \\
\textbf{2020-08-12}

1. Evaluate the Legendre symbol $\Leg{13}{23}$ using

(a) Euler's criterion

\textit{Solution.}

\vspace{5mm}

(b) Gauss's lemma

\textit{Solution.}

\newpage

2. Using the Quadratic Reciprocity Law, show that
%
\begin{equation*}
    \Leg{3}{p}
    = (-1)^{\floor*{\frac{p + 1}{6}}}
    =
    \begin{cases}
        1, & p \equiv 1, 11 \Mod{12}, \\
        -1, & p \equiv 5, 7 \Mod{12}.
    \end{cases}
\end{equation*}

\textit{Solution.}

\newpage

4. For any prime $p$ of the form $4 k + 3$, prove that
$x^2 + \frac{p + 1}{4} \equiv 0 \Mod{p}$ has no solution.

\textit{Solution.}

\newpage

9. For which primes are there solutions $x^2 + y^2 \equiv 0 \Mod{p}$
with $(x, p) = (y, p) = 1$?

\textit{Solution.}

\newpage

10. For which prime powers are there solutions $x^2 + y^2 \equiv 0 \Mod{p^\alpha}$
with $(x, p) = (y, p) = 1$?

\textit{Solution.}

\newpage

11. Evaluate each of the following Jacobi or Legendre symbols:

(a) $\Leg{111}{1001}$

\textit{Solution.}

\vspace{5mm}

(d) $\Leg{105}{1009}$

\textit{Solution.}

\end{document}

