% Set up the document
\documentclass{article}

% Page size
\usepackage[
    letterpaper,]{geometry}

% Lines between paragraphs
\setlength{\parskip}{\baselineskip}
\setlength{\parindent}{0pt}

% Math
\usepackage{mathtools}
\usepackage{amssymb}
\usepackage{amsthm}
\usepackage{commath}

% Number sets
\newcommand{\C}{\mathcal{C}}
\newcommand{\N}{\mathbb{N}}
\newcommand{\Q}{\mathbb{Q}}
\newcommand{\R}{\mathbb{R}}
\newcommand{\Z}{\mathbb{Z}}

% Links
\usepackage{hyperref}

% Page numbers at top right
\usepackage{fancyhdr}
\pagestyle{fancy}
\fancyhf{}
\fancyhead[R]{\thepage}
\renewcommand\headrulewidth{0pt}

\begin{document}

\textbf{MATH 342 assignment 1} \\
\textbf{Matt Wiens \#301294492} \\
\textbf{2020-05-25}

1. Show that $16 n^4 + 4 n^2 + 1$ is composite for every $n \in \N$.

\begin{proof}

Note that we can express
%
\begin{equation}
    16 n^4 + 4 n^2 + 1 = (4 n^2 - 2 n + 1) (4 n^2 + 2 n + 1)
    .
    \label{q1-1}
\end{equation}
%
For all $n \in N$, both
$1 < 4 n^2 - 2 n + 1$ and
$1 < 4 n^2 + 2 n + 1$
(both polynomials are strictly increasing and have minimum values of $5$
and $7$, respectively). Also, both factors in~\eqref{q1-1} are clearly
strictly less than $16 n^4 + 4 n^2 + 1$, using~\eqref{q1-1} and the fact
that $16 n^4 + 4 n^2 + 1$ is always positive.

Hence we have that
%
\begin{equation*}
    16 n^4 + 4 n^2 + 1 = p q
\end{equation*}
%
for every $n \in \N$ where $1 < p, q < 16 n^4 + 4 n^2 + 1$.
Thus it follows that $16 n^4 + 4 n^2 + 1$ is composite for every $n \in \N$.

\end{proof}

\newpage

2. Prove that for any positive integer $n$ there is a prime factor of
   $n! + 1$ exceeding $n$.

\begin{proof}

We will consider two (exhaustive) cases. For the first case, suppose $n!
+ 1$ is prime. Then $n! + 1$ has a prime factor (namely, itself) which
is larger than $n$.

For the second case, suppose $n! + 1$ is composite. Then there exists
some prime factor $p \in \N$; that is, $p|n! + 1$ and $p$ is prime. If
$p > n$ then we are done. Suppose $p \leq n$: then clearly $p|n!$, but
taking this together with $p|n! + 1$ implies that $p|1$, which further
implies that $p = 1$---a contradiction. Hence we must have $p > n$.

\end{proof}

\newpage

3. Using question 2, show that there are infinitely many primes.

\begin{proof}

Suppose for contradiction that there were finitely many primes. Let $n$
be the largest prime. Then, by question 2, there exists a prime factor
$p$ of $n! + 1$ such that $p > n$. But this contradicts that $n$ was the
largest prime. Hence there must be infinitely many primes.

\end{proof}

\newpage

4. Given a positive $k \geq 2$ there exists a string of $k$ consecutive
   composite integers.

\begin{proof}

We will prove that the integers in the set $S = \cbr{m + (k + 1)!: m \in
\cbr{2, \ldots, k + 1}}$ form a string of $k$ consecutive composite
integers. It should be clear that the elements in $S$ can be arranged in
a string of $k$ consecutive integers. What remains is to show that each
element in $S$ is composite.

Each element of $S$ has the form $m + (k + 1)!$ where $2 \leq m \leq k +
1$. Clearly both $m|m$ and $m|(k + 1)!$, so $m|m + (k + 1)!$. Because $m
\neq 1, m + (k + 1)!$, it follows that $m + (k + 1)!$ is composite.
Hence every element of $S$ is composite.

\end{proof}

\newpage

5. Let $a, b, c, d \in \Z$ with $a, c \neq 0$. Prove that if $a|b$ and
   $c|d$ then $ac|(ad + bc)$.

\begin{proof}

Suppose both $a|b$ and $c|d$. Then for some $k_1, k_2 \in \Z$, we have
$a = k_1 b$ and $c = k_2 d$. Multiplying through by $c$ and $a$,
respectively, we also have $a c = k_1 (b c)$ and $a c = k_2 (a d)$. Thus
$ac|bc$ and $ac|ad$, so $ac|ad + bc$.

\end{proof}

\newpage

6. Find the greatest common divisor $d$ of $405$ and $126$ and integers
   $x$ and $y$ which satisfy $405x + 126y = d$.

\textit{Solution.}
We can find the greatest common divisor using the Euclidean Algorithm.
For the forwards step we have
%
\begin{align*}
    405 &= 3 \cdot 126 + 27, \\
    126 &= 4 \cdot 27 + 18, \\
    27 &= 1 \cdot 18 + 9, \\
    18 &= 2 \cdot 9 + 0.
\end{align*}
%
Hence the greatest common divisor $d = 9$. We can find the Bezout
coefficients by performing the backwards step:
%
\begin{align*}
    9 &= 27 - 1 \cdot 18 \\
      &= \del{405 - 3 \cdot 126} - 1 \cdot \del{126 - 4 \cdot 27} \\
      &= \del{405 - 3 \cdot 126} - 1 \cdot \del{126 - 4 \cdot \del{405 - 3 \cdot 126}} \\
      &= (1 + 4) 405 + (-3 - 1 - 12) 126 \\
      &= 5 \cdot 405 + (-16) \cdot 126
      .
\end{align*}
%
Hence the Bezout coefficients $x, y$ satisfying $405 x + 126 y = d$ are $x = 5, y = -16$.

\end{document}
