% Set up the document
\documentclass{article}

% Page size
\usepackage[
    letterpaper,]{geometry}

% Lines between paragraphs
\setlength{\parskip}{\baselineskip}
\setlength{\parindent}{0pt}

% Math
\usepackage{mathtools}
\usepackage{amssymb}
\usepackage{amsthm}
\usepackage{commath}

% Number sets
\newcommand{\C}{\mathcal{C}}
\newcommand{\N}{\mathbb{N}}
\newcommand{\Q}{\mathbb{Q}}
\newcommand{\R}{\mathbb{R}}
\newcommand{\Z}{\mathbb{Z}}

% Links
\usepackage{hyperref}

% Page numbers at top right
\usepackage{fancyhdr}
\pagestyle{fancy}
\fancyhf{}
\fancyhead[R]{\thepage}
\renewcommand\headrulewidth{0pt}

\begin{document}

\textbf{MATH 342 assignment 2} \\
\textbf{Matt Wiens \#301294492} \\
\textbf{2020-06-01}

1. Find the simple continued fraction expansion, not terminating with
   the partial quotient of $1$, of $503/126$.

\textit{Solution.}
Here we will generate the continued fraction expansion from the
Euclidean Algorithm:
%
\begin{equation*}
    504 = 4 \cdot 126 + 0
    .
\end{equation*}
%
Hence, we see that we can write the desired simple continued fraction
expansion as $503 / 126 = 4$ or, using the notation in the notes, $[4]$.

\newpage

2. Find the simple continued fraction of $\sqrt{7}$.

\textit{Solution.}

\newpage

3. Find all solutions where $x$ and $y$ are integers to the Diophantine
   equation
%
\begin{equation*}
    \frac{1}{x} + \frac{1}{y} = \frac{1}{5}
    .
\end{equation*}

\textit{Solution.}

\newpage

4. How many ways can change be made for one dollar, using each of the following coins?

(a) dimes and quarters;

\textit{Solution.}

\vspace{5mm}

(b) nickels, dimes, and quarters;

\textit{Solution.}

\vspace{5mm}

(c) pennies, nickels, dimes, and quarters.

\textit{Solution.}

\newpage

5. Let $[x]$ denote the greatest integer less than or equal to $x$. Show that

(a) $[\alpha + n] = [\alpha] + n$;

\begin{proof}

Here I will assume that $\alpha \in \R$ and $n \in \Z$ (these aren't
specified in the question but I think it's suggested).

Let $x \in \Z$ be the integer such that $x = [\alpha]$. Then we have
that $x \leq \alpha < x + 1$. Adding through by $n$ we have $x + n \leq
\alpha + n < x + n + 1$. This shows that $[\alpha + n] = x + n = [\alpha
+ n]$.

\end{proof}

\vspace{5mm}

(b) $\frac{m + 1}{n} \leq \sbr{\frac{m}{n}} + 1, n > 0$.

\begin{proof}

Here I will assume that $m, n \in \Z$. Let $x = \sbr{\frac{m}{n}}$. We
know that $x \leq \frac{m}{n} < x + 1$. Multiplying through by $n$ we
have that
%
\begin{equation}
    x n \leq m < x n + n
    .
    \label{eq:5b}
\end{equation}
%
We also have that $x n \leq m \leq x n + n - 1$: this is because if $n =
1$, $x = \frac{m}{n} = m$ so the inequality reduces to $m \leq m \leq
m$; if $n > 1$ then $n - 1 \geq 1$, so $m$ being an integer combined
with~\eqref{eq:5b} results in the desired inequality.

Dividing through by $n$, we get
$x \leq \frac{m}{n} \leq x + 1 - \frac{1}{n}$. Adding through by
$\frac{1}{n}$ we obtain
%
\begin{equation*}
    \frac{m + 1}{n} \leq x + 1 = \sbr{\frac{m}{n}} + 1
    .
\end{equation*}

\end{proof}

\end{document}
