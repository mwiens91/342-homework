% Set up the document
\documentclass{article}

% Page size
\usepackage[
    letterpaper,]{geometry}

% Lines between paragraphs
\setlength{\parskip}{\baselineskip}
\setlength{\parindent}{0pt}

% Math
\usepackage{mathtools}
\usepackage{amssymb}
\usepackage{amsthm}
\usepackage{commath}

% Number sets
\newcommand{\C}{\mathcal{C}}
\newcommand{\N}{\mathbb{N}}
\newcommand{\Q}{\mathbb{Q}}
\newcommand{\R}{\mathbb{R}}
\newcommand{\Z}{\mathbb{Z}}

% Links
\usepackage{hyperref}

% Page numbers at top right
\usepackage{fancyhdr}
\pagestyle{fancy}
\fancyhf{}
\fancyhead[R]{\thepage}
\renewcommand\headrulewidth{0pt}

\begin{document}

\textbf{MATH 342 assignment 2} \\
\textbf{Matt Wiens \#301294492} \\
\textbf{2020-06-01}

1. Find the simple continued fraction expansion, not terminating with
   the partial quotient of $1$, of $503/126$.

\textit{Solution.}
Here we will generate the continued fraction expansion from the
Euclidean Algorithm:
%
\begin{equation*}
    504 = 4 \cdot 126 + 0
    .
\end{equation*}
%
Hence, we see that we can write the desired simple continued fraction
expansion as $503 / 126 = 4$ or, using the notation in the notes, $[4]$.

\newpage

2. Find the simple continued fraction of $\sqrt{7}$.

\textit{Solution.}

Recognizing that $\sbr[1]{\sqrt{7}} = 2$ we can start by writing
%
\begin{equation*}
    \sqrt{7} = 2 + (\sqrt{7} - 2) = 2 + \frac{1}{\frac{1}{\sqrt{7} - 2}}
    .
\end{equation*}
%
We can simplify the denominator by multiplying by $1$:
%
\begin{equation*}
    \frac{1}{\sqrt{7} - 2} \cdot \frac{\sqrt{7} + 2}{\sqrt{7} + 2}
    = \frac{\sqrt{7} + 2}{3}
    .
\end{equation*}
%
Since $\sbr{\frac{\sqrt{7} + 2}{3}} = [1.55] = 1$, we can further write
%
\begin{equation*}
    \frac{\sqrt{7} + 2}{3} = 1 + \frac{\sqrt{7} - 1}{3}
    .
\end{equation*}
%
Hence we have that
%
\begin{equation*}
    \sqrt{7} = 2 + \frac{1}{1 + \frac{1}{\frac{3}{\sqrt{7} - 1}}}
    .
\end{equation*}
%
For the ``inner-most fraction'' we again multiply by 1:
%
\begin{equation*}
    \frac{3}{\sqrt{7} - 1} \cdot \frac{\sqrt{7} + 1}{\sqrt{7} + 1} = \frac{\sqrt{7} + 1}{2}
    .
\end{equation*}
%
Because $\sbr{\frac{\sqrt{7} + 1}{2}} = [1.82] = 1$, we can write
%
\begin{equation*}
    \frac{\sqrt{7} + 1}{2} = 1 + \frac{\sqrt{7} - 1}{2}
    .
\end{equation*}
%
Therefore, we can write our original expression as
%
\begin{equation*}
    \sqrt{7} = 2 + \frac{1}{1 + \frac{1}{1 + \frac{1}{\frac{2}{\sqrt{7} - 1}}}}
    .
\end{equation*}
%
Now we play the same game of multiplying by $1$ for the inner-most fraction:
%
\begin{equation*}
    \frac{2}{\sqrt{7} - 1} \cdot \frac{\sqrt{7} + 1}{\sqrt{7} + 1} = \frac{\sqrt{7} + 1}{3}
    .
\end{equation*}
%
Since $\sbr{\frac{\sqrt{7} + 1}{3}} = [1.22] = 1$ we can write
%
\begin{equation*}
    \frac{\sqrt{7} + 1}{3} = 1 + \frac{\sqrt{7} - 2}{3}
    ,
\end{equation*}
%
which lets us write our original expression as
%
\begin{equation*}
    \sqrt{7} = 2 + \frac{1}{1 + \frac{1}{1 + \frac{1}{1 + \frac{1}{\frac{3}{\sqrt{7} - 2}}}}}
    .
\end{equation*}
%
Now we multiply the inner-most fraction by $1$:
%
\begin{equation*}
    \frac{3}{\sqrt{7} - 2} \cdot \frac{\sqrt{7} + 2}{\sqrt{7} + 2} = \sqrt{7} + 2
    .
\end{equation*}
%
Since $\sbr[1]{\sqrt{7} + 2} = [4.65] = 4$, we can write
%
\begin{equation*}
    \sqrt{7} + 2 = 4 + (\sqrt{7} - 2)
    .
\end{equation*}
%
Coming back to our main expression we have
%
\begin{equation*}
    \sqrt{7} = 2 + \frac{1}{1 + \frac{1}{1 + \frac{1}{1 + \frac{1}{4 + \frac{1}{\frac{1}{\sqrt{7} - 2}}}}}}
    .
\end{equation*}
%
However, we have already solved for $\frac{1}{\sqrt{7} - 2}$ above, so
the pattern now repeats and we can conclude that
%
\begin{equation*}
    \sqrt{7} = [2; 1, 1, 1, 4, 1, 1, 1, 4, \ldots]
    .
\end{equation*}

\newpage

3. Find all solutions where $x$ and $y$ are integers to the Diophantine
   equation
%
\begin{equation*}
    \frac{1}{x} + \frac{1}{y} = \frac{1}{5}
    .
\end{equation*}

\textit{Solution.}
We first write this equation as $5x + 5y - xy = 0$. Solving for $x$ in
this equation we obtain
%
\begin{equation*}
    x = \frac{5 y}{y - 5}
    .
\end{equation*}
%
Carrying out this division, we further have
%
\begin{equation*}
    x = 5 + \frac{25}{y - 5}
    .
\end{equation*}
%
Thus our solutions are integral whenever $(y - 5)|25$; that is, when $y
- 5 = \pm 1, \pm 5, \pm 25$. This gives us the possible values for $y$
as $y = -20, 4, 6, 10, 30$ (making note that $x, y \neq 0$) which gives
us the ordered pairs
%
\begin{equation*}
    (x, y) \in \cbr{(4, -20), (-20, 4), (30, 6), (10, 10), (6, 30)}
    .
\end{equation*}

\newpage

4. How many ways can change be made for one dollar, using each of the following coins?

(a) dimes and quarters;

\textit{Solution.}
Let $d$ be the number of dimes and $q$ the number of quarters. Then we
have the equation $10 d + 25 q = 100$ where we constrain $d$ and $q$ to
be positive integers. By inspection we can see that $\gcd(10, 25) = 5$
and that $(d_0, q_0) = (10, 0)$ is a particular solution. Thus, all of
our solutions are given by
%
\begin{align*}
    d &= 10 + \frac{25}{5}n = 10 + 5k, \\
    q &= 0 - \frac{10}{5}n = -2 k
    ,
\end{align*}
%
with $k \in \Z$. Given are integral constraints on $d$ and $q$ we can
see that there are three ways we can solve this problem (corresponding
to $k = -2, -1, 0$).

\vspace{5mm}

(b) nickels, dimes, and quarters;

\textit{Solution.}
Here again $d$ and $q$ will have the same meaning, and we will introduce
the variable $n$ to be the number of nickels. Our equation for this part
is $5n + 10d + 25q = 100$. If we let $w = n + 2 d$ then we can solve the
equation $5w + 25q = 100$. By inspection we have that $\gcd(5, 25) = 5$
and that $(w_0, q_0) = (0, 4)$ is a particular solution. Hence we have
the solutions
%
\begin{align*}
    w &= 5 k_1, \\
    q &= 4 - k_1
    ,
\end{align*}
%
with $k_1 = 0, 1, 2, 3, 4,$. Now we can solve for $n + 2 d = 5 k_1$. We
can see by inspection that $\gcd(1, 2) = 1$ and that
$(n_0, d_0) = (5k_1, 0)$ is a particular solution, so solutions are
%
\begin{align*}
    n &= 5 k_1  + 2 k_2, \\
    d &= - k_2
    ,
\end{align*}
%
where $k_2 \in \Z$.

For $k_1 = 0$ we have one solution corresponding to $k_2 = 0$.
For $k_1 = 1$ we have three solutions corresponding to
$k_2 = -2, -1, 0$. For $k_1 = 2$ we have six solutions corresponding to
$k_2 = -5, \ldots, 0$. For $k_1 = 3$ we have eight solutions corresponding to
$k_2 = -7, \ldots, 0$. For $k_1 = 4$ we have eleven solutions corresponding to
$k_2 = -10, \ldots, 0$.

Hence in total we have $1 + 3 + 6 + 8 + 11 = 29$ solutions.

\vspace{5mm}

(c) pennies, nickels, dimes, and quarters.

\textit{Solution.}
Here we will make the observation that pennies must come in groups of
$5$ in the solution to this problem, and in this exact sense pennies
(more accurately, groups of $5$ pennies) are equivalent to nickels. In
part (b) the number of nickels in each solution was, in order,
%
\begin{equation*}
    0, 1, 3, 5, 0, 2, 4, 6, 8, 10, 1, 3, 5, 7, 9, 11, 13, 15, 0, 2, 4, 6, 8, 10, 12, 14, 16, 18, 20
    .
\end{equation*}
%
In each of these solutions, every nickel can be substituted for a group
of $5$ pennies. So if we have $x$ elements corresponding to $\$0.05$, each
element can be either a nickel or a penny. From combinatorics (this is
the ``stars and bars'' problem), there are
%
\begin{equation*}
    \binom{x + 2 - 1}{x} = x + 1
\end{equation*}
%
ways of dividing $x$ elements of $\$0.05$ into groups of $5$ pennies and
nickels. Hence the number of solutions to this problem is given by
%
\begin{align*}
    &1
    + 2
    + 4
    + 6
    + 1
    + 3
    + 5
    + 7
    + 9
    + 11
    + 2
    + 4
    + 6
    + 8
    + 10
    \\
    &+ 12
    + 14
    + 16
    + 1
    + 3
    + 5
    + 7
    + 9
    + 11
    + 13
    + 15
    + 17
    + 19
    + 21
    = 242
    .
\end{align*}
%
Note that there may be a non-tedious way of solving this directly using
number theory but this was not taught in lectures nor is it discussed in
the textbook, so I prefer using the combinatorics result for this part.

\newpage

5. Let $[x]$ denote the greatest integer less than or equal to $x$. Show that

(a) $[\alpha + n] = [\alpha] + n$;

\begin{proof}

Here I will assume that $\alpha \in \R$ and $n \in \Z$ (these aren't
specified in the question but I think it's suggested).

Let $x \in \Z$ be the integer such that $x = [\alpha]$. Then we have
that $x \leq \alpha < x + 1$. Adding through by $n$ we have $x + n \leq
\alpha + n < x + n + 1$. This shows that $[\alpha + n] = x + n = [\alpha
+ n]$.

\end{proof}

\vspace{5mm}

(b) $\frac{m + 1}{n} \leq \sbr{\frac{m}{n}} + 1, n > 0$.

\begin{proof}

Here I will assume that $m, n \in \Z$. Let $x = \sbr{\frac{m}{n}}$. We
know that $x \leq \frac{m}{n} < x + 1$. Multiplying through by $n$ we
have that
%
\begin{equation}
    x n \leq m < x n + n
    .
    \label{eq:5b}
\end{equation}
%
We also have that $x n \leq m \leq x n + n - 1$: this is because if $n =
1$, $x = \frac{m}{n} = m$ so the inequality reduces to $m \leq m \leq
m$; if $n > 1$ then $n - 1 \geq 1$, so $m$ being an integer combined
with~\eqref{eq:5b} results in the desired inequality.

Dividing through by $n$, we get
$x \leq \frac{m}{n} \leq x + 1 - \frac{1}{n}$. Adding through by
$\frac{1}{n}$ we obtain
%
\begin{equation*}
    \frac{m + 1}{n} \leq x + 1 = \sbr{\frac{m}{n}} + 1
    .
\end{equation*}

\end{proof}

\end{document}
