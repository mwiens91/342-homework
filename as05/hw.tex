% Set up the document
\documentclass{article}

% Page size
\usepackage[
    letterpaper,]{geometry}

% Lines between paragraphs
\setlength{\parskip}{\baselineskip}
\setlength{\parindent}{0pt}

% Math
\usepackage{mathtools}
\usepackage{amssymb}
\usepackage{amsthm}
\usepackage{commath}

% Number sets
\newcommand{\C}{\mathcal{C}}
\newcommand{\N}{\mathbb{N}}
\newcommand{\Q}{\mathbb{Q}}
\newcommand{\R}{\mathbb{R}}
\newcommand{\Z}{\mathbb{Z}}

% Links
\usepackage{hyperref}

% Page numbers at top right
\usepackage{fancyhdr}
\pagestyle{fancy}
\fancyhf{}
\fancyhead[R]{\thepage}
\renewcommand\headrulewidth{0pt}

\begin{document}

\textbf{MATH 342 assignment 5} \\
\textbf{Matt Wiens \#301294492} \\
\textbf{2020-07-20}

1. Show that if $n$ is a positive integer, then
%
\begin{equation*}
    \del[4]{\sum_{d|n} \tau(d)}^2 = \sum_{d|n} \tau(d)^3
    .
\end{equation*}

\textit{Solution.}
Here we use a result from analysis that for any $n \in \N$,
%
\begin{equation}
    \sum_{k = 1}^n k^3 = \del[4]{\sum_{k = 1}^n k}^2
    \label{eq:1-cheatcode}
    .
\end{equation}
%
Note that since $\tau$ is a multiplicative function, so is $\tau^3$.
Hence the summatory function $\sum_{d|n} \tau(d)^3$ is multiplicative.
We also have that because $\sum_{d|n} \tau(d)$ is a multiplicative
function, so is its square. Thus if we can show that the result holds
for arbitrary prime powers $p^\alpha$, then the result holds for all
$n \in \N$.

For all prime powers $n = p^\alpha$ we have
%
\begin{align*}
    \sum_{d|n} \tau(d)^3
        &= \sum_{k = 0}^\alpha \tau(p^k)^3 \\
        &= \sum_{k = 0}^\alpha (1 + k)^3 \\
        &= \del[4]{\sum_{d|n} (1 + k)}^2 \\
        &= \del[4]{\sum_{d|n} \tau(p^k)}^2 \\
        &= \del[4]{\sum_{d|n} \tau(d)}^2
        ,
\end{align*}
%
where in the third line we used~\eqref{eq:1-cheatcode}. Thus
the main result holds for all prime powers, and, because both
functions involved in the main result are multiplicative, it holds for
all $n \in \N$.

\newpage

2. Let $\sigma_k(n)$ denote the sum of the $k$th powers of the divisors
of $n$, so that
%
\begin{equation*}
    \sigma_k(n) = \sum_{d|n} d^k
    .
\end{equation*}
%
(a) Find $\sigma_3(12)$.

\textit{Solution.}
Here we have
%
\begin{equation*}
    \sigma_3(12)
        = \sum_{d|12} d^3
        = 1^3 + 2^3 + 3^3 + 4^3 + 6^3 + 12^3
        = 2044
        .
\end{equation*}

\vspace{5mm}

(b) Give a formula for $\sigma_k(p)$, where $p$ is prime.

\textit{Solution.}
If $p$ is a prime, then any $k$ we have
%
\begin{equation*}
    \sigma_k(p)
        = \sum_{d|p} d^k
        = 1^k + p^k
        = 1 + p^k
    .
\end{equation*}

\vspace{5mm}

(c) Give a formula for $\sigma_k(p^a)$, where $p$ is prime
and $a$ is a positive integer.

\textit{Solution.}
In this case, we have
%
\begin{equation*}
    \sigma_k(p^a)
        = \sum_{d|p^a} d^k
        = \sum_{j = 0}^a (p^j)^k
        = \sum_{j = 0}^a (p^k)^j
        = \frac{p^{k(a + 1)} - 1}{p^k - 1}
    .
\end{equation*}

\vspace{5mm}

(d) Show that the function $\sigma_k$ is multiplicative.

\textit{Solution.}
Since $f(n) = n$ is multiplicative, so is $f^3(n) = n^3$. Since
$\sigma_k$ is the summatory function of $f^3$, we have that $\sigma_k$
is also multiplicative.

\newpage

3. Prove that $\sum_{d|n} \phi(d) = n$.

\textit{Solution.}
The proof we use here closely follows the proof from the course textbook.
Let $n \in \N$. We will partition the integers $m = 1, \ldots, n$ into
classes $C_d$, where $m$ is a member of $C_d$ if and only if
$(m, n) = d$, or, equivalently, $(m / d, n / d) = 1$. The size of $C_d$
gives the number of natural numbers $m \leq n / d$ that are relatively
prime to $n / d$. Thus $|C_d| = \phi(n / d)$, and hence, because we
partitioned $m = 1, \ldots, n$ into the classes $C_d$,
%
\begin{equation*}
    n = \sum_{d | n} \phi(n / d)
    .
\end{equation*}
%
Because $\{n / d: d \mid n\} = \{d: d \mid n\}$, we also have
%
\begin{equation*}
    n = \sum_{d | n} \phi(d)
    .
\end{equation*}

\newpage

4. For how many consecutive integers can the Möbius function $\mu(n)$
take the value $0$?

\textit{Solution.}
There are infinitely many arbitrary long strings of such consecutive integers.
This can be shown easily using the Chinese Remainder Theorem. We will
show that for any $k$, there is a string of $k$ consecutive integers
such that the Möbius function is zero (that is, the string of integers
is not squarefree). For any set of $k$ distinct primes
$\{p_1, p_2, \ldots, p_k\}$, the set
of numbers $\{n + i: 0 \leq i \leq k - 1\}$ where $n$ satisfies
%
\begin{align*}
    n &\equiv 0 \mod p_1^2, \\
    n + 1 &\equiv 0 \mod p_2^2, \\
    &\ \ \vdots \\
    n + k - 1 &\equiv 0 \mod p_k^2
    ,
\end{align*}
%
is a string of $k$ non-squarefree integers. Since $k$ was arbitrary
(and since there are infinitely many primes), it follows that there are
infinitely many arbitrary long strings of non-squarefree consecutive integers.

\newpage

5. For how many consecutive integers can the Möbius function $\mu(n)$
take a non-zero value?

\textit{Solution.}
Let $n \in \N$ and consider the set $\{n, n + 1, n + 2, n + 3\}$.
At least one of the numbers in the set must be divisible by $4$ (this
can be seen clearly by considering congruences),
and hence the set of consecutive integers is not squarefree.
Thus we can have at most $3$ squarefree consecutive integers.
To show that such a string exists, we have $\{1, 2, 3\}$ is consecutive
and squarefree.

\newpage

6. Show that
%
\begin{equation*}
    \sum_{d|n} \mu^2(d) = 2^{\omega(n)}
    ,
\end{equation*}
%
where $\omega(n)$ gives the number of prime divisors of $n$.

\textit{Solution.}
Because $\mu$ is multiplicative, so is $\mu^2$; so is its
summatory function $\sum_{d|n} \mu^2(d)$. Thus if we can show that the
result holds for arbitrary prime powers $p^\alpha$, then the result
holds for all $n \in \N$.

For all prime powers $n = p^\alpha$ we have
%
\begin{equation*}
    \sum_{d|p^\alpha} \mu^2(d)
        = \sum_{k = 0}^\alpha \mu^2(p^k)
        = \mu^2(1) - \mu^2(p)
        = 1^2 + (-1)^2
        = 2
        = 2^{\omega(p^\alpha)}
    ,
\end{equation*}
%
which completes the proof.

\newpage

7. Let $f$ be a multiplicative with $f(1) = 1$. Show that
%
\begin{equation*}
    \sum_{d|n} \mu(d) f(d) = \prod_{i = i}^k (1 - f(p_i))
    ,
\end{equation*}
%
where $n = \prod_{i = i}^k p_i^{\alpha_i}$.

\textit{Solution.}
Because $f$ and $\mu$ are multiplicative functions, so is $\mu f$;
hence the summatory function $\sum_{d|n} \mu(d) f(d)$ is also
multiplicative. Thus if we can show that the result holds
for arbitrary prime powers $p^\alpha$, then the result holds for all
$n \in \N$.

For all prime powers $n = p^\alpha$ we have
%
\begin{equation*}
    \sum_{d|p^\alpha} \mu(d) f(d)
        = \sum_{k = 0}^\alpha \mu(p^k) f(p^k)
        = f(1) - f(p)
        = 1 - f(p)
    ,
\end{equation*}
%
which completes the proof.

\end{document}
