% Set up the document
\documentclass{article}

% Page size
\usepackage[
    letterpaper,]{geometry}

% Lines between paragraphs
\setlength{\parskip}{\baselineskip}
\setlength{\parindent}{0pt}

% Math
\usepackage{mathtools}
\usepackage{amssymb}
\usepackage{amsthm}
\usepackage{commath}

% Number sets
\newcommand{\C}{\mathcal{C}}
\newcommand{\N}{\mathbb{N}}
\newcommand{\Q}{\mathbb{Q}}
\newcommand{\R}{\mathbb{R}}
\newcommand{\Z}{\mathbb{Z}}

% Links
\usepackage{hyperref}

% Page numbers at top right
\usepackage{fancyhdr}
\pagestyle{fancy}
\fancyhf{}
\fancyhead[R]{\thepage}
\renewcommand\headrulewidth{0pt}

\begin{document}

\textbf{MATH 342 assignment 4} \\
\textbf{Matt Wiens \#301294492} \\
\textbf{2020-07-13}

1. How many incongruent solutions are there to the congruence
$x^5 + x - 6 \equiv 0 (\bmod 144)$?

\textit{Solution.}

\newpage

2. Show that every composite Fermat number $F_m = 2^{2^m} + 1$ is a
pseudoprime to the base $2$.

\textit{Solution.}

\newpage

3. Explain why we should not choose primes $p$ and $q$ that are too close
together to form the modulus $n$ in the $RSA$ cryptosystem. In particular,
show that using a pair of twin primes for $p$ and $q = p + 2$ would be
disastrous. (Use Fermat's factorization method.)

\textit{Solution.}

\newpage

4. Use the Pollard $p - 1$ factorization method to find a non-trivial
divisor of $6994241$.

\textit{Solution.}

\newpage

5. Find the primes $p$ and $q$ if $n = p q = 14647$ and $\phi(n) = 14400$.

\textit{Solution.}

\newpage

6. If the ciphertext message produced by RSA encryption with the key
$(e, n) = (13, 2747)$ is $22060755043611651737$, what is the plaintext
message?

\textit{Solution.}

\end{document}
