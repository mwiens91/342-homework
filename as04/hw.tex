% Set up the document
\documentclass{article}

% Page size
\usepackage[
    letterpaper,]{geometry}

% Lines between paragraphs
\setlength{\parskip}{\baselineskip}
\setlength{\parindent}{0pt}

% Math
\usepackage{mathtools}
\usepackage{amssymb}
\usepackage{amsthm}
\usepackage{commath}

% Operators
\newcommand{\Mod}[1]{\ (\mathrm{mod}\ #1)}
\DeclarePairedDelimiter{\ceil}{\lceil}{\rceil}

% Number sets
\newcommand{\C}{\mathcal{C}}
\newcommand{\N}{\mathbb{N}}
\newcommand{\Q}{\mathbb{Q}}
\newcommand{\R}{\mathbb{R}}
\newcommand{\Z}{\mathbb{Z}}

% Links
\usepackage{hyperref}

% Page numbers at top right
\usepackage{fancyhdr}
\pagestyle{fancy}
\fancyhf{}
\fancyhead[R]{\thepage}
\renewcommand\headrulewidth{0pt}

\begin{document}

\textbf{MATH 342 assignment 4} \\
\textbf{Matt Wiens \#301294492} \\
\textbf{2020-07-13}

1. How many incongruent solutions are there to the congruence
$x^5 + x - 6 \equiv 0 \Mod{144}$?

\textit{Solution.}
Our goal is to determine how many incongruent solutions exit to
%
\begin{equation}
    x^5 + x - 6 \equiv 0 \Mod{144}
    \label{eq:1-0}
\end{equation}
%
Here we recognize that $144 = 2^4 3^2$. If we can find
the number of solutions to both
%
\begin{align}
    x^5 + x - 6 &\equiv 0 \Mod{2^4},
    \label{eq:1-1}
    \\
    x^5 + x - 6 &\equiv 0 \Mod{3^2},
    \label{eq:1-2}
\end{align}
%
then, for each pair of solutions to~\eqref{eq:1-1} and~\eqref{eq:1-2},
we can find a unique solution using the Chinese Remainder Theorem
to~\eqref{eq:1-0}. Hence the number of solutions to~\eqref{eq:1-0} is
given by multiplying the number of solutions to~\eqref{eq:1-1}
and~\eqref{eq:1-2} together.

We first solve~\eqref{eq:1-1} using Hansel's Lemma.
Let $f(x) = x^5 + x - 6$; then $f^\prime(x) = 5 x^4 + 1$. The solutions
to $f(x) \equiv 0 \Mod{2}$ are $x \equiv 0, x \equiv 1 \Mod{2}$.
For the first solution, we have $f^\prime(0) \equiv 1 \Mod 2$.
Thus, Hansel's Lemma guarantees this lifts up to a solution $\bmod \ 2^4$.
For the second solution, we have $f^\prime(1) \equiv 6 \equiv 0 \Mod 2$
and $f(1) \equiv -4 \equiv 0 \Mod{2^2}$. In this case, Hansel's Lemma
tells us that $x \equiv 1, x \equiv 3 \Mod 4$ are solutions to
$f(x) \equiv 0 \Mod{2^2}$.

Since we know that $x \equiv 0 \Mod 2$ lifts uniquely to a solution
modulo $2^4$, we will only focus on lifting the solutions
$x \equiv 1, x \equiv 3 \Mod 4$
of $f(x) \equiv 0 \Mod{2^2}$ to $f(x) \equiv 0 \Mod{2^3}$.
For the first solution we have $f^\prime(1) \equiv 0 \Mod 2$
and $f(1) \equiv -4 \equiv 6 \Mod{2^3}$. Hence we cannot lift the
first solution up modulo $2^3$. For the second solution,
we have $f^\prime(3) \equiv 406 \equiv 0 \Mod 2$
and $f(3) \equiv 240 \equiv 0 \Mod{2^3}$. Thus we have that
both $x \equiv 3, x \equiv 7 \Mod{2^3}$ are solutions to
$f(x) \equiv 0 \Mod{2^3}$.

We now focus on lifting these solutions up to solutions to
$f(x) \equiv 0 \Mod{2^4}$. For the first solution,
we again have $f^\prime(1) \equiv 0 \Mod 2$ and
and $f(3) \equiv 240 \equiv 0 \Mod{2^4}$. Thus we have that
both $x \equiv 3, x \equiv 11 \Mod{2^4}$ are solutions to
$f(x) \equiv 0 \Mod{2^4}$. For the second solution
we have have $f^\prime(7) \equiv 12006 \Mod 2$ and
and $f(7) \equiv 16808 \equiv 8 \Mod{2^4}$. Hence we cannot
lift the second solution up modulo $2^4$.

To summarize what we have so far, we have found that~\eqref{eq:1-1} has
$3$ solutions modulo $2^4$. We now focus on finding the number of solutions
to~\eqref{eq:1-2}. By inspection, the only solution to $f(x) \equiv 0 \Mod 3$
is $x \equiv 0 \Mod 3$. Since $f^\prime(3) \equiv 406 \equiv 1 \Mod 3$,
Hansel's Lemma tells us we can lift up this solution to a unique solution
modulo $3^2$.

Therefore we have $3$ solutions to~\eqref{eq:1-1} and $1$ solution
to~\eqref{eq:1-2}. Therefore there are $3$ incongruent solutions
to~\eqref{eq:1-0}.

\newpage

2. Show that every composite Fermat number $F_m = 2^{2^m} + 1$ is a
pseudoprime to the base $2$.

\textit{Solution.}
Here I will assume $m \in \Z^+$. Let $F_m= 2^{2^m} + 1$ such that $F_m$
is composite. Trivially $F_m \equiv 2^{2^m} + 1 \equiv 0 \Mod{F_m}$,
so $2^{2^m} \equiv -1 \Mod{F_m}$. Raising both sides of this congruence
to $2^{2^m - m}$ we have
%
\begin{align*}
    &(2^{2^m})^{2^{2^m - m}} \equiv (-1)^{2^{2^m - m}} \Mod{F_m} \\
    &\iff 2^{2^{m + 2^{2^m} - m}} \equiv 1 \Mod{F_m} \\
    &\iff 2^{2^{2^m}} \equiv 1 \Mod{F_m} \\
    &\iff 2^{F_m - 1} \equiv 1 \Mod{F_m}
    .
\end{align*}
%
This shows that $F_m$ is a pseudoprime base $2$.

\newpage

3. Explain why we should not choose primes $p$ and $q$ that are too close
together to form the modulus $n$ in the RSA cryptosystem. In particular,
show that using a pair of twin primes for $p$ and $q = p + 2$ would be
disastrous. (Use Fermat's factorization method.)

\textit{Solution.}
If primes $p$ and $q$ are too close together then
we can express $n = pq = p (p + \Delta)$
where $\Delta$ is small by assumption. Then, given that
%
\begin{equation*}
    t = \ceil{\sqrt{n}}
    ,
\end{equation*}
%
we can use Fermat's Factorization to find squares of the form
%
\begin{equation*}
    t^2 - n, (t + 1)^2 - n, \ldots, (t + i)^2 - n, \ldots
    .
\end{equation*}
%
Once this is solved for some $i = m$, we have $(t + m)^2 - n = y^2$
and equivalently $(t + m + y)(t + m - y) = n$. Recognizing that
$p = t + m - y$ and $q = t + m + y$, we have $\Delta = 2 y$.
Hence the solution can be written
$(t + m - \Delta / 2)^2 - n = (\Delta / 2)^2$. Because $\Delta$ is small
and $t$ is close to $n$, $m$ is also small, and hence the solution is
reached quickly.

For the specific case of twin primes, suppose $p$ and $p + 2$ are twin
primes such that $n = p(p + 2)$. This case can be solved quickly because
here $\Delta = 2$ and so we only need to solve $(t + m - 1)^2 - n = 1$
for $m$, which is the quickest solution to solve given the above discussion.

\newpage

4. Use the Pollard $p - 1$ factorization method to find a non-trivial
divisor of $6994241$.

\textit{Solution.}
Here we will use the factorization method with $a = 2$. Then
%
\begin{align*}
    2^{2!} &\equiv 4 \Mod{6994241}, \\
    2^{3!} &\equiv 4^3 \equiv 64 \Mod{6994241}, \\
    2^{4!} &\equiv 64^4 \equiv 2788734 \Mod{6994241}, \\
    2^{5!} &\equiv 2788734^5 \equiv 3834705 \Mod{6994241}, \\
    2^{6!} &\equiv 3834705^6 \equiv 513770 \Mod{6994241}, \\
    2^{7!} &\equiv 513770^7 \equiv 443653 \Mod{6994241}, \\
    2^{8!} &\equiv 443653^8 \equiv 4355857 \Mod{6994241}, \\
    2^{9!} &\equiv 4355857^9 \equiv 6234656 \Mod{6994241}, \\
    2^{10!} &\equiv 6234656^{10} \equiv 6980798 \Mod{6994241}, \\
    2^{11!} &\equiv 6980798^{11} \equiv 2816519 \Mod{6994241}, \\
    2^{12!} &\equiv 2816519^{12} \equiv 5858224 \Mod{6994241}, \\
    2^{13!} &\equiv 5858224^{13} \equiv 1 \Mod{6994241}
    .
\end{align*}
%
We then compute $\gcd(5858224^{14} - 1, 6994241) = 3361$, which is prime.
Calculating $6994241 / 3361 = 2081$, we see that $6994241 = 3381 \cdot 2081$
where both factors are prime.

\newpage

5. Find the primes $p$ and $q$ if $n = p q = 14647$ and $\phi(n) = 14400$.

\textit{Solution.}
Since the Euler phi function is multiplicative, we have
%
\begin{equation*}
    \phi(n) = (p - 1) (q - 1) = p q - p - q + 1 = n - p - q + 1
\end{equation*}
%
Then, solving for $p$, we have $p = n - q + 1 - \phi(n)$. Substituting
this into $n = p q$ we have
%
\begin{equation*}
    n = (n - q + 1 - \phi(n)) q
    .
\end{equation*}
%
Rearranging this into a polynomial in $q$, we have
%
\begin{equation*}
    q^2 + (\phi(n) - n - 1) q + n = 0
    .
\end{equation*}
%
This has solutions
%
\begin{equation*}
    q = - \frac{\phi(n) - n - 1}{2} \pm \frac{1}{2} \sqrt{(\phi(n) - n - 1)^2 - 4 n}
    .
\end{equation*}
%
For our specific values of $n$ and $\phi(n)$, we have
%
\begin{align*}
    q &= - \frac{14400 - 14647 - 1}{2} \pm \frac{1}{2} \sqrt{(14400 - 14647 - 1)^2 - 4 \cdot 14647} \\
      &= 124 \pm 27 \\
      &= 151 \text{ or } 97
    .
\end{align*}
%
Both of these primes, and it turns out that $14647 = 151 \cdot 97$, so we can take
$p = 151$, $q = 97$.

\newpage

6. If the ciphertext message produced by RSA encryption with the key
$(e, n) = (13, 2747)$ is $22060755043611651737$, what is the plaintext
message?

\textit{Solution.}
Since $n$ is small, factorization is simple using a computer.
Using Python to test all primes $p$ satisfying
$2 \leq p \leq \sqrt{n} = 52.4$, we find that $41 \vert n$, and thus
$n = 41 \cdot 67$. Now that we have $p$ and $q$, we only need to find
$d$ to recover the private key. To start, $\phi(n) = 40 \cdot 66 = 2640$.
Then we compute $d$ using $13 d \equiv 1 \Mod{2640}$.

Using the Extended Euclidean Algorithm, we want to solve $13d - 2640 k = 1$:
%
\begin{align*}
    -2640 &= 13 \cdot (-203) + 12 \\
    13 &= 12 \cdot 1 + 1\\
    12 &= 12 \cdot 1
    ,
\end{align*}
%
so $d \equiv 2437 \Mod{2640}$. Thus the private key is $(41, 67, 2437)$.

Now we can decrypt the message. Here I will make the natural assumption
that the chunk size is $4$. Then for each of the blocks we have
%
\begin{align*}
    (2206)^{2437} &\equiv 617 \Mod{2747}, \\
    (0755)^{2437} &\equiv 404 \Mod{2747}, \\
    (0436)^{2437} &\equiv 1908 \Mod{2747}, \\
    (1165)^{2437} &\equiv 1306 \Mod{2747}, \\
    (1737)^{2437} &\equiv 1823 \Mod{2747}
    .
\end{align*}
%
Thus the plain text message is $0617 \ 0404 \ 1908 \ 1306 \ 1823$. Using
the table in the notes, this corresponds to GREETINGSX.

\end{document}
