% Set up the document
\documentclass{article}

% Page size
\usepackage[
    letterpaper,]{geometry}

% Lines between paragraphs
\setlength{\parskip}{\baselineskip}
\setlength{\parindent}{0pt}

% Math
\usepackage{mathtools}
\usepackage{amssymb}
\usepackage{amsthm}
\usepackage{commath}

% Operators
\newcommand{\Mod}[1]{\ (\mathrm{mod}\ #1)}

% Number sets
\newcommand{\C}{\mathcal{C}}
\newcommand{\N}{\mathbb{N}}
\newcommand{\Q}{\mathbb{Q}}
\newcommand{\R}{\mathbb{R}}
\newcommand{\Z}{\mathbb{Z}}

% Links
\usepackage{hyperref}

% Page numbers at top right
\usepackage{fancyhdr}
\pagestyle{fancy}
\fancyhf{}
\fancyhead[R]{\thepage}
\renewcommand\headrulewidth{0pt}

\begin{document}

\textbf{MATH 342 assignment 4} \\
\textbf{Matt Wiens \#301294492} \\
\textbf{2020-07-13}

1. How many incongruent solutions are there to the congruence
$x^5 + x - 6 \equiv 0 \Mod{144}$?

\textit{Solution.}
Our goal is to determine how many incongruent solutions exit to
%
\begin{equation}
    x^5 + x - 6 \equiv 0 \Mod{144}
    \label{eq:1-0}
\end{equation}
%
Here we recognize that $144 = 2^4 3^2$. If we can find
the number of solutions to both
%
\begin{align}
    x^5 + x - 6 &\equiv 0 \Mod{2^4},
    \label{eq:1-1}
    \\
    x^5 + x - 6 &\equiv 0 \Mod{3^2},
    \label{eq:1-2}
\end{align}
%
then, for each pair of solutions to~\eqref{eq:1-1} and~\eqref{eq:1-2},
we can find a unique solution using the Chinese Remainder Theorem
to~\eqref{eq:1-0}. Hence the number of solutions to~\eqref{eq:1-0} is
given by multiplying the number of solutions to~\eqref{eq:1-1}
and~\eqref{eq:1-2} together.

We first solve~\eqref{eq:1-1} using Hansel's Lemma.
Let $f(x) = x^5 + x - 6$; then $f^\prime(x) = 5 x^4 + 1$. The solutions
to $f(x) \equiv 0 \Mod{2}$ are $x \equiv 0, x \equiv 1 \Mod{2}$.
For the first solution, we have $f^\prime(0) \equiv 1 \Mod 2$.
Thus, Hansel's Lemma guarantees this lifts up to a solution $\bmod \ 2^4$.
For the second solution, we have $f^\prime(1) \equiv 6 \equiv 0 \Mod 2$
and $f(1) \equiv -4 \equiv 0 \Mod{2^2}$. In this case, Hansel's Lemma
tells us that $x \equiv 1, x \equiv 3 \Mod 4$ are solutions to
$f(x) \equiv 0 \Mod{2^2}$.

Since we know that $x \equiv 0 \Mod 2$ lifts uniquely to a solution
modulo $2^4$, we will only focus on lifting the solutions
$x \equiv 1, x \equiv 3 \Mod 4$
of $f(x) \equiv 0 \Mod{2^2}$ to $f(x) \equiv 0 \Mod{2^3}$.
For the first solution we have $f^\prime(1) \equiv 0 \Mod 2$
and $f(1) \equiv -4 \equiv 6 \Mod{2^3}$. Hence we cannot lift the
first solution up modulo $2^3$. For the second solution,
we have $f^\prime(3) \equiv 406 \equiv 0 \Mod 2$
and $f(3) \equiv 240 \equiv 0 \Mod{2^3}$. Thus we have that
both $x \equiv 3, x \equiv 7 \Mod{2^3}$ are solutions to
$f(x) \equiv 0 \Mod{2^3}$.

We now focus on lifting these solutions up to solutions to
$f(x) \equiv 0 \Mod{2^4}$. For the first solution,
we again have $f^\prime(1) \equiv 0 \Mod 2$ and
and $f(3) \equiv 240 \equiv 0 \Mod{2^4}$. Thus we have that
both $x \equiv 3, x \equiv 11 \Mod{2^4}$ are solutions to
$f(x) \equiv 0 \Mod{2^4}$. For the second solution
we have have $f^\prime(7) \equiv 12006 \Mod 2$ and
and $f(7) \equiv 16808 \equiv 8 \Mod{2^4}$. Hence we cannot
lift the second solution up modulo $2^4$.

To summarize what we have so far, we have found that~\eqref{eq:1-1} has
$3$ solutions modulo $2^4$. We now focus on finding the number of solutions
to~\eqref{eq:1-2}. By inspection, the only solution to $f(x) \equiv 0 \Mod 3$
is $x \equiv 0 \Mod 3$. Since $f^\prime(3) \equiv 406 \equiv 1 \Mod 3$,
Hansel's Lemma tells us we can lift up this solution to a unique solution
modulo $3^2$.

Therefore we have $3$ solutions to~\eqref{eq:1-1} and $1$ solution
to~\eqref{eq:1-2}. Therefore there are $3$ incongruent solutions
to~\eqref{eq:1-0}.

\newpage

2. Show that every composite Fermat number $F_m = 2^{2^m} + 1$ is a
pseudoprime to the base $2$.

\textit{Solution.}
Here I will assume $m \in \Z^+$. Let $F_m= 2^{2^m} + 1$ such that $F_m$
is composite. Trivially $F_m \equiv 2^{2^m} + 1 \equiv 0 \Mod{F_m}$,
so $2^{2^m} \equiv -1 \Mod{F_m}$. Raising both sides of this congruence
to $2^{2^m - m}$ we have
%
\begin{align*}
    &(2^{2^m})^{2^{2^m - m}} \equiv (-1)^{2^{2^m - m}} \Mod{F_m} \\
    &\iff 2^{2^{m + 2^{2^m} - m}} \equiv 1 \Mod{F_m} \\
    &\iff 2^{2^{2^m}} \equiv 1 \Mod{F_m} \\
    &\iff 2^{F_m - 1} \equiv 1 \Mod{F_m}
    .
\end{align*}
%
This shows that $F_m$ is a pseudoprime base $2$.

\newpage

3. Explain why we should not choose primes $p$ and $q$ that are too close
together to form the modulus $n$ in the RSA cryptosystem. In particular,
show that using a pair of twin primes for $p$ and $q = p + 2$ would be
disastrous. (Use Fermat's factorization method.)

\textit{Solution.}

\newpage

4. Use the Pollard $p - 1$ factorization method to find a non-trivial
divisor of $6994241$.

\textit{Solution.}

\newpage

5. Find the primes $p$ and $q$ if $n = p q = 14647$ and $\phi(n) = 14400$.

\textit{Solution.}

\newpage

6. If the ciphertext message produced by RSA encryption with the key
$(e, n) = (13, 2747)$ is $22060755043611651737$, what is the plaintext
message?

\textit{Solution.}

\end{document}
