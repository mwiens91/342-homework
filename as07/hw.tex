% Set up the document
\documentclass{article}

% Page size
\usepackage[
    letterpaper,]{geometry}

% Lines between paragraphs
\setlength{\parskip}{\baselineskip}
\setlength{\parindent}{0pt}

% Math
\usepackage{mathtools}
\usepackage{amssymb}
\usepackage{amsthm}
\usepackage{commath}

% Operators
\newcommand{\Mod}[1]{\ (\mathrm{mod}\ #1)}
\DeclareMathOperator{\ord}{ord}

% Number sets
\newcommand{\C}{\mathcal{C}}
\newcommand{\N}{\mathbb{N}}
\newcommand{\Q}{\mathbb{Q}}
\newcommand{\R}{\mathbb{R}}
\newcommand{\Z}{\mathbb{Z}}

% Links
\usepackage{hyperref}

% Page numbers at top right
\usepackage{fancyhdr}
\pagestyle{fancy}
\fancyhf{}
\fancyhead[R]{\thepage}
\renewcommand\headrulewidth{0pt}

\begin{document}

\textbf{MATH 342 assignment 7} \\
\textbf{Matt Wiens \#301294492} \\
\textbf{2020-08-05}

1. Show that if $n$ is a positive integer, and $a$ and $b$ are integers
relatively prime to $n$ such that $(\ord_n a, \ord_n b) = 1$, then
$\ord_n ab = \ord_n a \cdot \ord_n b$.

\textit{Solution.}
Let $x \coloneqq \ord_n a$, $y \coloneqq \ord_n b$, and
$z \coloneqq \ord_n ab$. We will show that $x y \mid z$
and $z \mid x y$ to conclude that $x y = z$.

To see that $x y \mid z$ we will show that both $x \mid z$ and $y \mid
z$; since $(x, y) = 1$, it will follow that $x y \mid z$. First,
to show that $x \mid z$, consider that
%
\begin{equation*}
    1
    \equiv 1^y
    \equiv ((ab)^z)^y
    \equiv a^{y z} b^{y z}
    \equiv (b^y)^z a^{y z}
    \equiv 1^z a^{y z}
    \equiv a^{y z}
    \Mod{n}
    .
\end{equation*}
%
Hence, $x \mid yz$, but $(x, y) = 1$, so we can conclude $x \mid z$.
Similarly, to show that $y \mid z$, we consider
%
\begin{equation*}
    1
    \equiv 1^x
    \equiv ((ab)^x)^y
    \equiv a^{x z} b^{x z}
    \equiv (a^x)^z b^{x z}
    \equiv 1^z b^{x z}
    \equiv b^{x z}
    \Mod{n}
    ,
\end{equation*}
%
and, using a similar argument as before, we conclude that $y \mid z$.
Since $x \mid z$ and $y \mid z$, and $(x, y) = 1$, it follows that $x y \mid z$.

Now, to show that $z \mid xy$, consider that
%
\begin{equation*}
    (ab)^{xy}
    \equiv a^{xy} b^{xy}
    \equiv (a^x)^y (b^y)^x
    \equiv 1^y 1^x
    \Mod{n}
    ,
\end{equation*}
%
from which it follows that $z \mid xy$.

Since we have shown both $xy \mid z$ and $z \mid xy$, it follows that
$xy = z$, or, equivalently, $\ord_n ab = \ord_n a \cdot \ord_n b$.

\newpage

2. Show that if $n$ is a positive integer, and $a$ is relatively prime
to $n$, then $\ord_n \bar{a} = \ord_n a$, where $\bar{a}$ is the inverse
of $a$ modulo $n$.

\textit{Solution.}
Let $x \coloneqq \ord_n a$ and $y \coloneqq \ord_n \bar{a}$. Consider
%
\begin{align*}
    &a \bar{a} \equiv 1 \Mod{n} \\
    &\implies (a \bar{a})^x \equiv 1 \Mod{n} \\
    &\iff a^x \bar{a}^x \equiv 1 \Mod{n} \\
    &\iff \bar{a}^x \equiv 1 \Mod{n}.
\end{align*}
%
Hence $x \mid y$. We also have
%
\begin{align*}
    &a \bar{a} \equiv 1 \Mod{n} \\
    &\implies (a \bar{a})^y \equiv 1 \Mod{n} \\
    &\iff a^y \bar{a}^y \equiv 1 \Mod{n} \\
    &\iff a^y \equiv 1 \Mod{n}.
\end{align*}
%
so $y \mid x$. Thus $x = y$, or, equivalently, $\ord_n \bar{a} = \ord_n a$.

\newpage

3. (a) Let $k$ and $a$ be positive integers, with $a \geq 2$. Show
that $k \mid \phi(a^k - 1)$.

\textit{Solution.}
We first show that $\ord_{a^k - 1} a = k$. To see this, consider that
for all $0 \leq j < k$, $0 < a^j - 1 < a^k - 1$, and so
$a^j - 1 \not\equiv 0 \Mod{a^k - 1}$ for all $0 \leq j < k$.
That is, for all $0 \leq j < k$,
%
\begin{equation*}
    a^j \not\equiv 1 \Mod{a^k - 1}
    .
\end{equation*}
%
But, $a^k - 1 \equiv 0 \Mod{a^k - 1}$, so $a^k \equiv 1 \Mod{a^k - 1}$
and hence $\ord_{a^k - 1} a = k$.

Thus using that $a^{\phi(a^k - 1)} \equiv 1 \Mod{a^k - 1}$, we have that
$\ord_{a^k - 1} a = k \mid \phi(a^k - 1)$.

\vspace{5mm}

(b) Let $p$ be a prime number. Show that if $p \mid \phi(m)$ and
$p \nmid m$, then there is at least one prime factor $q$ of $m$
such that $q \equiv 1 \Mod{p}$.

\textit{Solution.}
Here I will assume that $m$ is a positive integer. Suppose the prime
factorization of $m$ is $m = \prod_{k = 1}^n p_k^{a_k}$. Then we have
that
%
\begin{align*}
    \phi(m)
    &= \prod_{k = 1}^n (p_k^{a_k} - p_k^{a_k - 1}) \\
    &= \prod_{k = 1}^n p_k^{a_k - 1} (p_k - 1) \\
    &= \del[4]{\prod_{k = 1}^n p_k^{a_k - 1}} \del[4]{\prod_{k = 1}^n (p_k - 1)}
    .
\end{align*}
%
Now, since $p \nmid m$, we have that for all $k = 1, \ldots, n$, $p \nmid p_k$.
Therefore, $p \nmid \prod_{k = 1}^n p_k^{a_k - 1}$, and so we must have that
$p \mid \prod_{k = 1}^n (p_k - 1)$ given that $p \mid \phi(m)$. Since $p$
is a prime, we further have that for some $k = 1, \ldots, n$, $p \mid p_k - 1$.
Thus assigning $q \coloneqq p_k$, a prime, we have that $p \mid q - 1$, and hence
$q \equiv 1 \Mod{p}$. Hence there exists at least one prime factor $q$ of $m$
such that $q \equiv 1 \Mod{p}$.

\vspace{5mm}

(c) Let $p$ be a given prime number. Prove that there exists infinitely
many prime numbers $q \equiv 1 \Mod{p}$.

\textit{Solution.}
Here we will use parts (a) and (b) of this question. Suppose for contradiction
that there are only many $n$ primes $q_k$ satisfying $q_k \equiv 1 \Mod{p}$
for $k = 1, \ldots, n$. Let $a \coloneqq p \prod_{k = 1}^n q_k$. Then, using part (a)
we have that $p \mid \phi(a^p - 1)$. Since $p \mid a^p$, $p \nmid a^p - 1$.
Hence according to part (b) there is a prime factor $q$ of $a^p - 1$
such that $q \equiv 1 \Mod{p}$. However, for all $k = 1, \ldots, n$, $q_k \nmid a^k - 1$
(since $q_k \mid a^k$). Thus for all $k$, $q \neq q_k$. This contradicts our assumption
that the only primes satisfying satisfying $q \equiv 1 \Mod{p}$
were $q_k$ with $k = 1, \ldots, n$.

Hence, there must be infinitely many prime numbers of the form $q \equiv 1 \Mod{p}$.
\newpage

4. For which positive integers $a$ is the congruence
$a x^4 \equiv 2 \Mod{13}$ solvable?

\textit{Solution.}
Note that $a x^4 \equiv 2 \Mod{13}$ is equivalent to
$x^4 \equiv \bar{a} 2 \Mod{13}$. Using the notation used in lectures,
let $m = 13$ and $k = 4$.
Then we have that $\phi(m) = 12$ and that $d \coloneqq (k, \phi(m)) =  4$.
Using the theorem from lectures and that
$a x^4 \equiv 2 \Mod{13}$ is equivalent to
$x^4 \equiv \bar{a} 2 \Mod{13}$, we have that $a$ permits a solution to
$a x^4 \equiv 2 \Mod{13}$ provided that
%
\begin{equation*}
    (\bar{a} 2)^{\phi(m) / d} \equiv (\bar{a} 2)^3 \equiv 8 \bar{a}^3 \equiv 1 \Mod{13}
    .
\end{equation*}
%
The inverses modulo $13$ of $a = 1, \ldots, 12$ are shown in the table below
as well as the value $8 \bar{a}^3$ modulo $13$.

\begin{table}[!ht]
\centering
\begin{tabular}{| c | c | c |}
 \hline
 $a$ & $\bar{a}$ & $\bmod(8 \bar{a}^3, 13)$ \\
 \hline
 1 & 1 & 8 \\
 2 & 7 & 1 \\
 3 & 9 & 8 \\
 4 & 10 & 5 \\
 5 & 8 & 1 \\
 6 & 11 & 1 \\
 7 & 2 & 12 \\
 8 & 5 & 12 \\
 9 & 3 & 8\\
 10 & 4 & 5 \\
 12 & 12 & 5 \\
 \hline
\end{tabular}
\end{table}

Hence we have that all positive integers $a$ congruent modulo $13$ to $5$ or $6$
permit a solution to $a x^4 \equiv 2 \Mod{13}$.

\end{document}

